\PoemTitle{Lilie}

\begin{verse}
Umřela panna v~době jarních let, \\
jako když uschne mladé růže květ; \\
umřela panna, růže v~poupěti -- \\
škoda jí, škoda v~zemi ležeti!
\end{verse}

\begin{verse}
„Nedávejte mne ve vsi na hřbitov, \\
tam bývá nářek sirotků a vdov, \\
tam slzí hořkých mnoho plynulo: \\
srdéčko mé by hořem hynulo.
\end{verse}

\begin{verse}
Pochovejte mne vpod zelený les, \\
tam na mém hrobě kvésti bude vřes; \\
ptáčkové mi tam budou zpívati: \\
srdéčko moje bude plesati."
\end{verse}

\begin{verse}
Neminul ještě ani rok a den, \\
hrob její drobným vřesem povlečen; \\
nepřišlo ještě ani do tří let, \\
na jejím hrobě vzácný květe květ.
\end{verse}

\begin{verse}
Lilie bílá -- kdo ji uviděl, \\
každého divný pojal srdce žel; \\
lilie vonná -- kdo ji pocítil, \\
v~každém se touhy plamen roznítil. --
\end{verse}

\begin{verse}
„Hoj, moje chaso, vraného mi stroj! \\
Chce mi se na lov pod zelenou chvoj, \\
chce mi se na lov pod jedlový krov: \\
zdá mi se, dnes že vzácný bude lov!"
\end{verse}

\begin{verse}
Halohou, halou! v~chrtů poštěkot, \\
příkop nepříkop -- hop! a plot neplot: \\
pán na vraníku napřaženou braň \\
a jako šipka před ním bílá laň.
\end{verse}

\begin{verse}
„Halohou, halou! vzácná moje zvěř, \\
nespasí tebe pole ani keř!" \\
Zdviženo rámě, jež ji probije -- \\
tu místo laňky -- bílá lilie.
\end{verse}

\begin{verse}
Pán na lilii hledí s~údivem, \\
rámě mu kleslo, duch se tají v~něm; \\
myslí a myslí -- prsa dmou se výš, \\
vůní či touhou? Kdo mu rozumíš?
\end{verse}

\begin{verse}
„Hoj, sluho věrný, ku práci se měj: \\
tu lilii mi odtud vykopej; \\
v~zahradě své chci tu lilii mít -- \\
zdá mi se, bez ní že mi nelze byt!
\end{verse}

\begin{verse}
„Hoj, sluho věrný, důvěrníče můj, \\
tu lilii mi střež a opatruj, \\
opatruj mi ji pilně v~den i noc -- \\
divná, podivná k~ní mě pudí moc!"
\end{verse}

\begin{verse}
Opatroval ji jeden, druhý den; \\
pán její vnadou divně přeblažen. \\
Leč noci třetí, v~plné luny svit, \\
pospíchá sluha pána probudit.
\end{verse}

\begin{verse}
„Vstávej, pane můj, chyba v~odkladě: \\
tvá lilie se vláčí po sadě; \\
pospěš, nemeškej, pravýť nyní čas: \\
tvá lilie si divný vede hlas!"
\end{verse}

\begin{verse}
„Životem vratkým smutná živořím, \\
co v~poli rosa, co na řece dým: \\
jasně slunečný svitne paprsek -- \\
rosa i pára, i můj zhyne věk!"
\end{verse}

\begin{verse}
„Nezhyne věk tvůj, tuť důvěru mám; \\
před sluncem jistou ochranu ti dám: \\
zdi pevné budou tvojí záštitou, \\
ač, duše milá, budeš chotí mou."
\end{verse}

\begin{verse}
Vdala se za něj; blaze bydlila, \\
až i synáčka jemu povila. \\
Pán hody slaví, štěstí svého jist; \\
tu mu královský posel nese list.
\end{verse}

\begin{verse}
„Můj věrný milý!" tak mu píše král, \\
„chci, abys zejtra ke službě mi stál; \\
chci, aby přijel každý věrný lech, \\
potřeba velká -- všeho doma nech."
\end{verse}

\begin{verse}
Smutně se loučil s~milou chotí svou, \\
jako by tušil svou nehodu zlou. \\
„A když mi strážcem nelze býti tvým, \\
svou matku tobě strážci zůstavím."
\end{verse}

\begin{verse}
Špatně mu matka vůli plnila, \\
špatně manželku jeho střežila; \\
na nebi slunce -- pobořena síň: \\
„Zhyn, paní noční, zhyn, obludo, zhyn!"
\end{verse}

\begin{verse}
Pán jede domů -- dosti služby jest; \\
tu mu žalostná v~ústrety jde věst: \\
„Tvé pacholátko již ti nežije \\
a po tvé paní -- zvadlá lilie!"
\end{verse}

\begin{verse}
„Ó matko, matko, ty hadice zlá, \\
čím ublížila tobě žena má? \\
Otrávila jsi žití mého květ: \\
bodejž i tobě zčernal boží svět!"
\end{verse}