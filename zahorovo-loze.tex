\PoemTitle{Záhořovo lože}
\begin{verse}
I~\end{verse}

\begin{verse}
Šedivé mlhy nad lesem plynou, \\
jako duchové vlekouce se řadem; \\
jeřáb ulétá v~krajinu jinou -- \\
pusto a nevlídno ladem i sadem. \\
Vítr od západu studeně věje \\
a přižloutlé listí tichou píseň pěje. \\
Známá tě to píseň; pokaždéť v~jeseni \\
listové na dubě šepcí ji znova: \\
ale málokdo pochopuje slova, \\
a kdo pochopí, do smíchu mu není.
\end{verse}

\begin{verse}
Poutníče neznámý v~hábitě šerém, \\
s~tím křížem v~ruce na dlouhé holi \\
a s~tím růžencem -- kdo jsi ty koli, \\
kam se ubíráš nyní podvečerem? \\
Kam tak pospícháš? Tvá noha bosa, \\
a jeseň chladná -- studená rosa: \\
zůstaň zde u~nás, jsme dobří lidi, \\
dobrého hosta každý rád vidí. --
\end{verse}

\begin{verse}
Poutníče milý! -- Ne tys ještě mladý, \\
ještě vous tobě nepokrývá brady \\
a tvoje líce jako pěkné panny -- \\
ale co tak bledé a smutně zvadlé, \\
a tvoje oči v~důlky zapadlé! \\
Snad je ve tvém srdci žel pochovaný? \\
Snad že neštěstí tvé tělo svíží \\
lety šedivými dolů k~zemi níží?
\end{verse}

\begin{verse}
Mládenče pěkný, nechoď za noci, \\
možné-li, budem rádi ku pomoci, \\
a přinejmenším snad potěšíme. \\
Jen nepomíjej, pojď, pohov tělu: \\
není bez léku nižádného želu \\
a mocný balzám v~důvěře dříme. --
\end{verse}

\begin{verse}
Nic neslyší, neví, ani oko zvedne, \\
není ho možné ze snů vytrhnouti! \\
A~tam již zachází v~chrastině jedné: \\
pánbůh ho posilň na jeho pouti!
\end{verse}

\begin{verse}
II
\end{verse}

\begin{verse}
Daleké pole, široké pole, \\
předlouhá cesta přes to pole běží, \\
a podle cesty pahorek leží, \\
a dřevo štíhlé stojí na vrchole: \\
štíhlá to jedlice -- však beze snětí, \\
jen malá příčka svrchu přidělána \\
a na té příčce přibitý viděti rozpjatý obraz Krista Pána. \\
Hlavu krvavou vpravo nakloňuje, \\
ruce probité roztahuje v~šíři: \\
v~dvě světa strany jimi ukazuje, \\
v~dvě strany protivné, jako cesta míří: \\
pravou na východ, kde se světlo rodí, \\
levou na západ, kde noc vojevodí. \\
Tam na východě nebeská je brána, \\
tam u~věčném ráji bydlí boží svatí; \\
a kdo dobře činí, čáka jemu dána, \\
že se tam s~nimi té bude radovati. \\
Ale na západě jsou pekelná vrata, \\
tam plane mořem síra i smola, \\
tam pletou ďáblové, zlá rota proklatá, \\
zlořečené duše v~ohnivá kola. \\
Vpravo, Kriste Pane, tam dej nám dospěti, \\
však od levice vysvoboď své děti!
\end{verse}

\begin{verse}
Tu na tom pahorku leže na kolenou \\
náš mladý poutník v~ranním světla kmitu, \\
okolo kříže ruku otočenou, \\
vroucně objímá dřevo beze citu. \\
Brzy cos šepce, slzy roně z~oka, \\
brzy zase vzdychá -- těžce, zhluboka. --
\end{verse}

\begin{verse}
Takto se loučí od své drahé panny \\
mládenec milý v~poslední době, \\
ubíraje se v~cizí světa strany, \\
ani pak věda, sejdou-li se k~sobě; \\
ještě poslední vroucí obejmutí, \\
ještě políbení jako plamen žhoucí: \\
již měj se tu dobře, dívko přežádoucí, \\
chvíle nešťastná pryč odtud mne nutí! --
\end{verse}

\begin{verse}
Tvář jako stěna, pohledění ledné, \\
ale v~srdci plamen zhoubný, divoký, \\
náhle se poutník ze země zvedne \\
a k~západu rychlé zaměří kroky. --
\end{verse}

\begin{verse}
Brzy potom zmizel v~hustém lesa proutí: \\
pánbůh ho potěš na jeho pouti!
\end{verse}

\begin{verse}
III
\end{verse}

\begin{verse}
Stojí, stojí skála v~hlubokém lese, \\
podle ní cesta v~habrovém houští, \\
a na té skále dub velikán pne se, \\
král věkovitý nad věčnou poutí: \\
k~nebesům holé vypínaje čelo, \\
zelená ramena drží na vše strany; \\
tuhý oděv jeho hromem rozoraný \\
a pod oděvem vyhnilé tělo: \\
dutina prostranná, příhodná velmi -- \\
pohodlný nocleh líté lesní šelmy!
\end{verse}

\begin{verse}
A~hle, pod tím dubem na mechovém loži \\
čí je ta postava veliká, hrozná? \\
Zvíře, či člověk v~medvědí koži? \\
Sotva kdo člověka v~tom stvoření pozná! \\
Tělo jeho -- skála na skále ležící, \\
údy jeho -- svaly dubového kmene, \\
vlasy a vousy vjedno splývající \\
s~ježatým obočím tváři začazené; \\
a pod obočím zrak bodající, \\
zrak jedovatý, podobný právě \\
zraku hadímu v~zelené trávě. \\
Kdo je ten člověk? A~to mračné čelo, \\
jakými obmysly se jest obestřelo? \\
Kdo je ten člověk? Co chce v~této poušti? --
\end{verse}

\begin{verse}
Nic se mne neptej, ohledni se v~houští \\
z~obé strany cesty; zeptej se těch kostí, \\
ježto tu leží práchnivějíce; \\
zeptej se těch černých, nevlídných hostí, \\
ježto tu krákají obletujíce: \\
ti mnoho viděli -- ti vědí více!
\end{verse}

\begin{verse}
Tu však muž lesní z~lože svého skočí, \\
zrak upřený v~cestu divoce plane; \\
kyjem ohromným nad hlavou točí \\
a skok za skokem prostřed cesty stane.
\end{verse}

\begin{verse}
Kdo přichází cestou? -- V~hábitě mládenec, \\
kříž maje v~ruce, za pasem růženec! -- \\
Utec, mládenče, obrať se zpátky! \\
Tvá cesta v~jistou tebe smrt uvádí.
\end{verse}

\begin{verse}
Životě lidský i beztoho krátký, \\
a škoda tvého panenského mládí! \\
Obrať se, utíkej, co ti síla stačí, \\
dokud kyj ohromný na tě nepřikvačí \\
a neroztříští tvou hlavičku v~kusy! --
\end{verse}

\begin{verse}
Neslyší, nevidí, v~želu svém hlubokém \\
jde dále před se povlovným krokem, \\
kde života svého pozbýti musí. --
\end{verse}

\begin{verse}
„Stůj, červe, kdo jsi, kam tě cesta vede?“
\end{verse}

\begin{verse}
Zastavil se poutník, zvedna líce bledé: \\
„Jsem zatracenec,“ -- odpovídá tiše -- \\
„do pekla cesta má, do satanské říše!“
\end{verse}

\begin{verse}
„Hoho, do pekla? -- Čtyřicáté léto, \\
co již tu sedím, mnoho jsem slyšíval, \\
mnoho vidíval, ale písně této \\
potud mi nikdo ještě nezazpíval! -- \\
Hoho, do pekla? Netřeba ti kroků, \\
sám tě tam dopravím, nevzdechneš ani! -- \\
Však a se naplní můj počet roků, \\
přijdu snad za tebou také na snídaní!“ --
\end{verse}

\begin{verse}
„Nic ty se nerouhej milosti boží! \\
Dříve než jsem viděl den života prvý, \\
zapsán jsem peklu otce svého krví \\
klamem ďábelským -- pro pozemské zboží. \\
Milost boží velká, a znamení kříže \\
zlámeť i strašlivé pekelné mříže, \\
porazí satana se vší jeho mocí! \\
Milost boží velká, ta ráčí dáti, \\
že se slabý poutník co vítěz navrátí, \\
dobuda zápisu z~pekelné noci.“ --
\end{verse}

\begin{verse}
„Co pravíš? -- Za těch let, za čtyřiceti, \\
bez počtu jsem jich do pekla sklátil, \\
však se ještě nikdo zpátky nenavrátil! -- \\
Slyš, červe, jsi mladé, heboučké pleti, \\
byl bys mi dobře, místo tuhé zvěři, \\
za malou pochoutku dnes na večeři; \\
ale pustím tě -- nechám tebe jíti -- \\
však ještě nikdo, co jich tu šlo koli, \\
neušel mojí sukovité holi! -- \\
Pustím tě, červe, ale to chci míti: \\
přisahej, že potom věrně mi povíš, \\
co v~pekle uvidíš a čeho se dovíš."
\end{verse}

\begin{verse}
I~vztýčil se poutník, a vysoko zdviže \\
hůl svou poutnickou se znamením kříže: \\
"Přisahám na kříže svatého slávu, \\
že ti z~pekla věrnou přinesu zprávu!“
\end{verse}

\begin{verse}
IV
\end{verse}

\begin{verse}
Minula zima, sníh na horách taje, \\
v~údolích povodeň od sněhu a deště; \\
jeřáb se vrací z~dalekého kraje, \\
avšak náš poutník nepřichází ještě.
\end{verse}

\begin{verse}
V~zeleň oděly se v~lese ratolesti, \\
fiala pode křem milou vůni dýše, \\
slavík vypravuje dlouhé pověsti, \\
ale žádná zpráva z~pekelné říše.
\end{verse}

\begin{verse}
Ušlo jaro -- léto; již se dnové krátí, \\
povětří chladne, listí opadává, \\
z~pekla však žádná, žádná nejde zpráva. \\
Zdali se poutník přece ještě vrátí? \\
Zdali nekleslo cestou jeho tělo? \\
Zdali ho peklo v~sobě nepozřelo? --
\end{verse}

\begin{verse}
Lesní muž pod dubem z~vysokého stanu \\
škaredě pohlíží v~západní stranu; \\
sedí a bručí: „Co jich tu šlo koli, \\
neušel nikdo mé sukovité holi! \\
Jen jednoho jsem na slovo vsadil, \\
jen toho jednoho -- a ten mne zradil!“ --
\end{verse}

\begin{verse}
„Ó nezradil tebe!“ -- ozval se týmž časem \\
podle muže poutník povýšeným hlasem; \\
postava přímá, oko přísné, smělé, \\
studený poklid na jeho čele \\
a z~jeho bledé, ušlechtilé tváře \\
jako by planula slunečná záře.
\end{verse}

\begin{verse}
„Nezradil jsem tebe, přísahou ztuha \\
zavázal jsem se ti, hříšný boží sluha; \\
a nyní přisahám ještě tobě znova: \\
přisahám na kříže svatého slávu, \\
že tobě věrnou z~pekla nesu zprávu!“
\end{verse}

\begin{verse}
Zachvěl se muž lesní, slyše tato slova, \\
a vyskočil vzhůru, sáhna po své zbroji; \\
avšak jako bleskem omráčený stojí -- \\
nesneseť zrak jeho zraku poutníkova.
\end{verse}

\begin{verse}
„Tuto seď a slyš, pověsti hrůzy \\
zvěstuji tobě po pekelné chůzi; \\
o~hněvu božím slovo moje svědčí, \\
ale milost božská neskončeně větší!“
\end{verse}

\begin{verse}
Vypravuje poutník, co v~pekle spatřil: \\
moře plamenův -- břidké ďáblův pluky; \\
a kterak se život s~věčnou smrtí sbratřil \\
na věčné, vždy nové zatracencův muky. --
\end{verse}

\begin{verse}
Muž lesní pod dubem zamračený sedí, \\
nemluví slova -- jen před sebe hledí. \\
Vypravuje poutník, co v~pekle slyšel: \\
úpěnlivé nářky -- zlořečené klení -- \\
volání pomoci -- však nikoho není, \\
kdo by tu potěšil, kdo k~pomoci přišel, \\
jen věčná kletba, věčné zatracení! --
\end{verse}

\begin{verse}
Muž lesní pod dubem zamračený sedí, \\
nemluví slova -- jen před sebe hledí.
\end{verse}

\begin{verse}
Vypravuje poutník, jak znamením kříže \\
přinutil satana, pekelné kníže, \\
rozkázati ďáblu, strůjci klamu zlého, \\
aby zase vrátil krví psanou blánu. \\
Protivil se ďábel pekelnému pánu, \\
nevrátil zápisu dle rozkazu jeho.
\end{verse}

\begin{verse}
Rozlítil se satan a v~zlosti své velí: \\
„Vykoupejte jeho v~pekelné koupeli!“ \\
Učinila rota dle jeho rozkazu, \\
připravila lázeň z~ohně a mrazu: \\
z~jedné strany hoří jako uhel vzňatý, \\
z~druhé strany mrzne v~kámen ledovatý; \\
a když vidí rota míru naplněnu, \\
obrací zmrzlinu opak do plamenů. \\
Strašlivě řve ďábel, jako had se svíjí, \\
a ho pak již smysl i cit pomíjí. \\
Tu pokynul satan, rota odstoupila, \\
a síla zas nová ďábla oživila. \\
Ale když propuštěn opět dýše lehce, \\
krví psané blány přec vydati nechce. --
\end{verse}

\begin{verse}
Rozlítiv se satan, káže ve svém hněvu: \\
„Nuže ať obejme pekelnou děvu!“ -- \\
A~byla ta děva z~železa skuta, \\
rámě vztažené k~toužebné milosti: \\
přivinula ďábla na svá prsa krutá, \\
a zdrceny jsou všecky jeho kosti. \\
Strašlivě řve ďábel, jako had se svíjí, \\
a ho pak již smysl i cit pomíjí. \\
Tu pokynul satan, panna povolila, \\
a síla zas nová ďábla oživila. \\
Ale když propuštěn opět dýše lehce, \\
krví psané blány přec vydati nechce. --
\end{verse}

\begin{verse}
I~zařičel satan poslední své slovo: \\
„Uvrzte jeho v~lože Záhořovo!“ --
\end{verse}

\begin{verse}
„V lože Záhořovo? V~Záhořovo lože?“ -- \\
volá v~uděšení muž divý v~lese, \\
hrozné tělo jeho osikou se třese \\
a pot vyráží z~tuhé čela kože. \\
„Lože Záhořovo! -- Záhoř je to jméno, \\
od matky mé někdy často vysloveno, \\
když učívala mne plésti rohože, \\
když mi rohožemi na mechu stlávala \\
a vlčí kožinou mne přikrývala. \\
A~nyní v~pekle Záhořovo lože -? \\
Však pověz mi ty -- ty sluho boží, \\
co čeká Záhoře na pekelném loži?“ --
\end{verse}

\begin{verse}
„Spravedliva jest pomsty boží ruka, \\
leč ukryto věčné jeho usouzení: \\
neznámá mi sice tvá pekelná muka, \\
ale tvých zločinů nic menší není. \\
Nebo věz, že ďábel, slyše ona slova, \\
zhroziv se pokuty lože Záhořova, \\
krvavý zápis vrátil bez prodlení!“ --
\end{verse}

\begin{verse}
Stoletá sosna na chlumové stráni \\
hrdě vypíná k~nebi své témě: \\
i přijde sekera, sosna hlavu sklání, \\
a těžkým pádem zachvěje se země.
\end{verse}

\begin{verse}
Divoký tur lesní v~bujnosti své síly \\
z~kořene vyvrací mocná v~lese dřeva; \\
proboden oštěpem, potácí se chvíli, \\
a padne, v~bolesti smrtelné řeva. --
\end{verse}

\begin{verse}
Takto muž lesní. Poražen tou zvěstí \\
na zemi klesá ve smrtelném strachu; \\
řve a svíjí se, bije v~hlavu pěstí, \\
nohy poutníkovy objímaje v~prachu: \\
„Smiluj se, pomoz, pomoz, muži boží, \\
nedej mi dospěti k~pekelnému loži!“ --
\end{verse}

\begin{verse}
„Nemluv takto ke mně, červ jsem, roveň tobě, \\
bez milosti boží ztracen věčně věkův: \\
k~ní ty se obrať, od ní prose lékův, \\
a čiň pokání v~pravé ještě době. --
\end{verse}

\begin{verse}
„Kterak mám se káti? Viz tu na mé holi \\
ty řady vrubův, co jich tu jest koli -- \\
spočti je, můžeš-li -- ta známka každá, \\
každý ten vroubek jest jedna vražda!“ --
\end{verse}

\begin{verse}
I~zvedne poutník, k~zemi se nakloně, \\
hůl Záhořovu -- kmen mocné jabloně -- \\
a zarazí ji v~tvrdé skály témě \\
jako tenký proutek do zorané země.
\end{verse}

\begin{verse}
„Tu kleč přede svědkem svých hrozných činů, \\
kleč ve dne v~noci, ukrutný zlosynu! \\
Času nepočítej, nedbej žízně, hladu, \\
jedno počítej svých zločinů řadu, \\
lituj a pros boha, aby smazal vinu. \\
Vina tvá jest velká, těká, bez příkladu: \\
bez příkladu budiž i tvoje pokání, \\
a bez konce jest boží smilování! \\
Tu kleč a čekej -- až se v~jedné době \\
z~milosti boží vrátím zase k~tobě.“
\end{verse}

\begin{verse}
Takto dí poutník, a jde cestou dále. -- \\
A~Záhoř klečí, klečí neustále; \\
klečí ve dne, v~noci -- nepije, nejí; \\
vzdychaje božího prosí smilování. -- \\
Den po dni mine; již i sníh se shání, \\
ledové mrazy jižjiž přicházejí; \\
a Záhoř klečí, prosit nepřestává -- \\
ale na poutníka darmo očekává, \\
ten nepřichází, nevrací se k~němu. \\
Bůh budiž milostiv muži kajícnému!
\end{verse}

\begin{verse}
V~\end{verse}

\begin{verse}
Devadesáte let přeletělo světem; \\
mnoho se zvrátilo zatím od té chvíle: \\
kdo onoho času býval nemluvnětem, \\
jest nyní starcem, do hrobu se chýle.
\end{verse}

\begin{verse}
Avšak málo jich dozrálo k~té době, \\
ostatní všickni jsou schováni v~hrobě. \\
Jiné pokolení -- cizí obličeje -- \\
vše ve světě cizí, kam se člověk děje; \\
jen to slunéčko modravého nebe, \\
jenom to nižádné proměny nevzalo; \\
a jako před věky lidi těšívalo, \\
tak i nyní ještě vždy blaží tebe!
\end{verse}

\begin{verse}
Jest opět jaro. Vlažný větřík duje, \\
na lukách svěží kolébá se tráva; \\
slavík své pověsti opět vypravuje \\
a fialka novou zas vůni vydává.
\end{verse}

\begin{verse}
Habrovým stínem hlubokého lesa \\
dvé poutníků se cestou ubírá: \\
shrbený stařeček, v~ruce berlu nesa, \\
berlu biskupskou, věkem již se třesa, \\
a pěkný mládenec, ten jej podepírá.
\end{verse}

\begin{verse}
„Posečkej, synu můj, rád bych odpočinul, \\
odpočinutí si má duše žádá!“ \\
Rád bych se ji k~otcům zesnulým přivinul, \\
ale milost boží jinak mi ukládá. \\
Milost boží velká, ta sluhu svého \\
mocně provedla skrz pekelnou bránu, \\
v~úřadě svém svatém povýšila jeho; \\
a proto duše má dobrořečí Pánu. \\
Pevně jsem doufal v~tebe, Hospodine: \\
dej, ať tvá sláva na zemi spočine! -- \\
Synu můj, žízním, ohledni se vůkol: \\
tuším, ač není-li mdlých smyslů mámení, \\
tuším, že mi blízké najdeš občerstvení, \\
aby byl dokonán můj života úkol.“
\end{verse}

\begin{verse}
Odešel mládenec, v~lesní zašel strany, \\
zdali by kde našel pramen uchovaný. \\
I~dere se houštím, kráčí dál a dále, \\
až i se prodere k~mechovité skále. \\
Ale tu náhle noha jeho stane \\
a jako světluška večer létající \\
leskne se podiv v~pěkné jeho líci: \\
divně neznámá vůně k~němu vane, \\
vůně nevýslovná, neskončené vnady, \\
jako by v~rajské vstupoval sady. \\
A~když pak mládenec skrze husté chvojí \\
vzhůru se prodere a na skálu vkročí, \\
věc nepodobnou vidí jeho oči: \\
na holé skále strom košatý stojí, \\
strom jabloňový, v~šíř se rozkládaje, \\
na něm ovoce divné krásy zraje -- \\
jablka zlatá -- a z~nich se nese \\
ta rajská vůně vůkol po všem lese.
\end{verse}

\begin{verse}
I~zplesalo srdce v~mládencovu těle \\
a zrak jeho čilý jiskřil se vesele: \\
„Ach jistě, jistě, bůh dobrotivý \\
stařečkovi k~vůli tu své činí divy: \\
pro posilu jemu -- místo chladné vody -- \\
pustá v~lese skála rajské nese plody.“ \\
Ale jak s~ochotou po jablku sáhne, \\
tak s~uleknutím ruku zas odtáhne.
\end{verse}

\begin{verse}
„Ty nech, netrhej -- však jsi nesázel!“ \\
hlas dutý, hluboký káže jemu ztuha, \\
blízký hlas, jako by ze země vycházel, \\
neb nikde vůkol neviděti druha. \\
Jen pařez veliký stojí vedle něho, \\
po něm ostružiny s~mechem se vinou, \\
a podál zbytky dubu prastarého, \\
kmen rozdrcený s~šírou vydutinou.
\end{verse}

\begin{verse}
Obešel jinoch peň, prohlíží dutinu, \\
obešel tu celou okolní krajinu: \\
však ani stopy nalézti nemoha, \\
že by tu kráčela kdy lidská noha, \\
všude jen pouhou viděti pustinu.
\end{verse}

\begin{verse}
„A snad se ucho mé obluzeno šálí? \\
Snad zvíře divoké zařvalo v~dáli? \\
Snad od vody v~skále zvuk onen pocházel,“ \\
dí k~sobě sám jinoch; a nedbaje zvuku, \\
opět po jablku vztahuje ruku.
\end{verse}

\begin{verse}
„Ty nech, netrhej -- však jsi nesázel!“ \\
hlas dutý hřmotněji zapovídá zase. \\
A~když se mládenec ohlédl po hlase, \\
hle, pařez veliký mezi ostružinou \\
hýbati se počne a z~mechu se šinou \\
dlouhá dvě ramena, k~jinochu měříce, \\
a nad rameny, jako smolné svíce \\
v~mlhavé noci, dvé červených očí \\
zpod šedého mechu k~němu se točí.
\end{verse}

\begin{verse}
Zděsil se mládenec a znamením kříže \\
znamená se jednou, po druhé a třetí, \\
a jak vyplašené z~hnízda ostříže, \\
nehledaje cesty, nevida obtíže, \\
přímo ze skály houštím dolů letí \\
a zkrvavený od ostrých snětí \\
na zemi padne k~stařečkovi blíže.
\end{verse}

\begin{verse}
„Ach pane, pane, zle je v~tomto lese: \\
košatá jabloň na skále, na pláni, \\
a jabloň na jaře zralé plody nese, \\
a pařez veliký trhati je brání. \\
A~ten pařez mluví, očima točí \\
a chytá ramenem, kdo k~jabloni kročí: \\
ach pane, ďáblovo tu jest panování!“
\end{verse}

\begin{verse}
„Mýlíš se, synu můj, tuto milost boží \\
své divy činí -- budiž jemu sláva! \\
Vidím, že pouť moje již se dokonává, \\
rádo se tělo mé v~zemi této složí! -- \\
Ještě mi posluž naposled, můj synu, \\
doveď mne nahoru, na skalnou planinu.“
\end{verse}

\begin{verse}
Učinil tak jinoch: napřed cestu klestí, \\
a potom stařečka po ní musil nésti. --
\end{verse}

\begin{verse}
A~když již přišli nahoru k~jabloni, \\
aj, tu se pařez ke stařečku kloní, \\
vztahuje rámě vstříc a raduje se: \\
„Ach pane, pane můj, dlouhos nepřicházel: \\
hle, tvá sazenice již ovoce nese, \\
ach utrhni, pane, však sám jsi sázel.“ --
\end{verse}

\begin{verse}
„Záhoři, Záhoři, pokoj budiž tobě: \\
pokoj ti přináším v~poslední své době! \\
Bez míry, bez konce jest milost boží, \\
nás oba vytrhla pekelnému loži! \\
Propusť mne nyní ji, jakož i já tebe: \\
nechť se tu popel náš vedle sebe složí, \\
a ducha nechť vezmou andělové z~nebe!“
\end{verse}

\begin{verse}
„Amen!“ dí Záhoř. A~v~tom okamžení \\
sesul se ve skrovnou prachu hromádku; \\
a jen ostružina na holém kamení \\
zůstala státi, jemu na památku.
\end{verse}

\begin{verse}
Zároveň i stařec mrtev na zem klesá -- \\
pouť jeho pozemská již dokonána! -- \\
I~zůstal mládenec sám uprostřed lesa, \\
by ještě vykonal vůli svého pána.
\end{verse}

\begin{verse}
Leč nad hlavou jeho té samé chvíle \\
vznášejí se dvě holubice bílé; \\
v~radostném plesu vznášejí se vzhůru, \\
až i se vznesly k~andělskému kůru.
\end{verse}