\PoemTitle{Věštkyně (úlomky)}

\begin{verse}
Když oko vaše slzou se zaleje, \\
když na vás těžký padne čas, \\
tehda přináším větvici naděje, \\
tu se můj věští ozve hlas.
\end{verse}

\begin{verse}
Nechtějte vážiti lehce řeči mojí, \\
z~nebeť přichází věští duch; \\
zákon nezbytný ve všem světě stojí \\
a vše tu svůj zaplatí dluh.
\end{verse}

\begin{verse}
Řeka si hledá konce svého v~moři, \\
plamen se k~nebi temeni; \\
co země stvoří, sama zase zboří: \\
avšak nic nejde v~zmaření.
\end{verse}

\begin{verse}
Jisté a pevné jsou osudu kroky, \\
co má se státi, stane se; \\
a co den jeden v~své pochová toky, \\
druhý zas na svět vynese.
\end{verse}

\begin{verse}
*
\end{verse}

\begin{verse}
Viděla jsem muže na Bělině vodě, \\
praotce slavných vojvodů, \\
an za svým pluhem po dědině chodě, \\
vzdělával země úrodu.
\end{verse}

\begin{verse}
Tu přišli poslové od valného sněmu, \\
a knížetem jest oráč zván, \\
oblekli oděv zlatoskvoucí jemu, \\
a nedoorán zůstal lán.
\end{verse}

\begin{verse}
Položil rádlo a propustil voly: \\
„Odkud jste vyšli, jděte zpět!" \\
a své bodadlo zarazil tu v~poli, \\
aby pučilo v~list i květ.
\end{verse}

\begin{verse}
Pojala voly nedaleká hora -- \\
podnes ji značí vody rmut; \\
a suchá holi lískovice kora \\
vydala trojí bujný prut.
\end{verse}

\begin{verse}
A~pruty vzkvětly a ovoce nesly: \\
leč dospěl jenom jich jeden; \\
druhé dva zvadly a ze stromu klesly, \\
nevzkřísivše se po ten den.
\end{verse}

\begin{verse}
Slyšte a vězte -- nejsouť marné hlasy, \\
vložte je pilně na paměť: \\
nastane doba, přijdou zase časy, \\
kdež obživne i mrtvá sněť.
\end{verse}

\begin{verse}
Obě ty větve v~ušlechtilém květu \\
vzmohou se šíře, široce \\
a nenadále ku podivu světu \\
přinesou blahé ovoce.
\end{verse}

\begin{verse}
Tu přijde kníže ve zlatě a nachu, \\
aby zaplatil starý dluh, \\
a vyndá na svět ze smetí a prachu \\
Přemyslův zavržený pluh.
\end{verse}

\begin{verse}
A~z~duté hory ven povolá voly \\
i zase k~pluhu přiděje \\
a zanedbanou doorá tu roli \\
a zlatým zrnem zaseje.
\end{verse}

\begin{verse}
I~vzejde setí, jaře bude kvésti, \\
bujně se zaskví zlatý klas: \\
a s~ním i vzejde země této štěstí \\
a stará sláva vstane zas.
\end{verse}

\begin{verse}
*
\end{verse}

\begin{verse}
Viděla jsem skálu nad řekou se pnoucí, \\
na skále Krokův zlatý hrad; \\
okolo hradu květnatí palouci -- \\
kněžny Libuše květný sad.
\end{verse}

\begin{verse}
Pod hradem dole staveníčko milé -- \\
kněžnina lázeň na řece; \\
viděla jsem kněžnu tváři ušlechtilé, \\
ve stříbroskvoucím obleče. \\
Na prahu stála zmilené své lázně, \\
patřila v~smutný říční proud; \\
četla tam slova naděje i bázně: \\
své milé země tajný soud.
\end{verse}

\begin{verse}
„Vidím požáry a krvavé boje, \\
ostrý meč tebe probode, \\
vidím tvou bídu, pohanění tvoje; \\
však nezoufej, můj národe!"
\end{verse}

\begin{verse}
Tu jí dvě panny, stojíce po boku, \\
zlatou kolébku podaly; \\
políbila ji a v~bezedném toku \\
ji potopila v~podskalí.
\end{verse}

\begin{verse}
Slyšte a vězte slova Libušina -- \\
já slyšela její věští hlas: \\
„Tuto spočívej, lůžko mého syna, \\
až povolám tě někdy zas!
\end{verse}

\begin{verse}
Z~temného lůna hlubokého moře \\
povstane nový, mladý svět; \\
široké lípy v~mém otcovském dvoře \\
ponesou zase vonný květ.
\end{verse}

\begin{verse}
Smutné osení vzkřísí těžký příval, \\
z~noci se zrodí jasný den: \\
a národ, kterýž prve slavný býval, \\
ten bude opět oslaven.
\end{verse}

\begin{verse}
Tehda na světlo ze propasti řeky \\
zlatá kolébka vyplyne, \\
a země spása, souzená před věky, \\
na ní co dítko spočine."
\end{verse}

\begin{verse}
*
\end{verse}

\begin{verse}
Viděla jsem tebe, lůžko blahosvaté, \\
znám tebe již, má hvězdo, znám, \\
však to mi ještě klidnou mysl mate, \\
kdy tebe opět uhlídám.
\end{verse}

\begin{verse}
Léto za letem bez ustání běží, \\
zima za zimou uhání: \\
důvěra má však nepohnutě leží \\
a co rok roste doufání.
\end{verse}

\begin{verse}
Když pak se létě z~hloubi pod skalinou \\
nejeden nenavrátí zpět \\
a když se zimě s~veselou družinou \\
prolomí pod saněmi led,
\end{verse}

\begin{verse}
vzdychávám: Ejhle, k~pluku Libušinu \\
přibylo nových oudů zas! \\
Kdy přijde doba, kdež si odpočinu? \\
Ach ještě není, není čas!
\end{verse}

\begin{verse}
Neb takto v~knihách osudu je psáno, \\
slyšte a vězte moji zvěst: \\
Až bude svítat ono blahé ráno, \\
vstanou i mrtví v~jeho čest.
\end{verse}

\begin{verse}
Tehda Libuše u~velikém pluku \\
své vojsko vodní postaví, \\
a vzhůru zvednouc mateřskou svou ruku, \\
svůj národ český oslaví!"
\end{verse}

\begin{verse}
Viděla jsem kostel nad Orlicí řekou, \\
slyšela jsem jeho zlatý zvon, \\
prv než upřímnost českou starověkou \\
rozerval lítých vášní shon.
\end{verse}

\begin{verse}
Když pak utuchly v~Čechách boží ctnosti: \\
víra i láska s~nadějí, \\
ukryl se kostel v~země hlubokosti \\
a vody místo stápějí.
\end{verse}

\begin{verse}
Však nezůstane věky věkův v~hrobě: \\
odtekouť jednou vody zas, \\
i vstane kostel v~bývalé ozdobě \\
a zazní slavně zvonu hlas.
\end{verse}

\begin{verse}
Slyšte a vězte, takto stojí psáno, \\
tak uloženo v~osudí: \\
„Tehdáž uzříte zlatých časů ráno, \\
až zlatý zvon je probudí.
\end{verse}

\begin{verse}
Až z~oné strany nad Orlici řeku \\
nový les vítr zaseje, \\
až mlází vzroste, až i svého věku \\
boroví toto dospěje \\
a toho lesa krajní borovice \\
až sama sebou dožije, \\
uschne a zetlíc padne do Orlice, \\
a již i kořen vyhnije:
\end{verse}

\begin{verse}
I~tehdáž vyryje přijdouc divá svině \\
poslední zbytky na průhon, \\
a tu pod nimi z~rumu v~rozvalině \\
zjeví se zase zlatý zvon.
\end{verse}

\begin{verse}
Neb tak určeno již od první chvíle: \\
by v~podzemní se vydal pouť \\
a ve svůj čas že vytčeného cíle \\
za řekou musí dosáhnouť." --
\end{verse}

\begin{verse}
Vězte, na stráni pod zeleným borem \\
kmen souzený již dozrává: \\
vysoký, mocný, beze větví skorém, \\
jen vršek čerstvý zůstává.
\end{verse}

\begin{verse}
Zdali zvon také již je na své pouti? \\
Zdali včas cíle dospěje? \\
Kdo zprávy jisté může poskytnouti \\
a posíliti naděje?
\end{verse}

\begin{verse}
Aj, viděla jsem, an tu sedlák oře \\
na poli blíže Bystřice, \\
svou ranní píseň zpívaje v~pokoře: \\
„Ó bože, svatá Trojice!"
\end{verse}

\begin{verse}
Tu divný odpor orání překazil, \\
z~brázdy se vymklo ruchadlo: \\
„Ký ďábel z~pekla mi tu co přimrazil? \\
Bodejž se i s~ním propadlo!"
\end{verse}

\begin{verse}
Tak oráč zaklel, a do prohlubení \\
zapadal pronikavý hlas -- \\
zlatého zvonu žalostné zaznění: \\
„Ach ještě není, není čas!"
\end{verse}

\begin{verse}
Ach ještě není, ještě čas tu není, \\
však přichyl ucho k~zemi blíž \\
a pod kořeny jedle z~dáli znění \\
zlatého zvonu uslyšíš.
\end{verse}

\begin{verse}
*
\end{verse}

\begin{verse}
Nenaříkejte, neštěstí a osud \\
že vás tak tvrdě potkaly, \\
však naříkejte, že jste jimi posud \\
rozumnější se nestali!
\end{verse}

\begin{verse}
Aj, vidím hóru nad jiné zvýšenou -- \\
hora ta dobře známa vám -- \\
bujnými sady kolem otočenou, \\
a na té hoře boží chrám.
\end{verse}

\begin{verse}
Avšak do chrámu branou chodí trojí \\
a trojí vycházejí zas; \\
slyšte a vězte, toto psáno stojí, \\
a v~srdci složte věští hlas:
\end{verse}

\begin{verse}
„Darmo nadějí kojíte se planou, \\
nezbudete svých psot a bíd, \\
dokavad jednou chodívati branou \\
nebude tvrdý český lid!"
\end{verse}

\begin{verse}
Kdos uši dostal, abyjima slyšel, \\
proč si je palcem zacpáváš? \\
A~komu rozum z~vysokosti přišel, \\
proč po něm nohou šlapáváš?
\end{verse}

\begin{verse}
Tisíc let ušlo, co své milé syny \\
svornosti učil Svatopluk, \\
však neproniknul dotud, do hodiny, \\
moudrého slova zlatý zvuk!
\end{verse}

\begin{verse}
*
\end{verse}

\begin{verse}
Vy, kdo znajíce otcův slavné činy, \\
jimi se rádi chlubíte; \\
tam na pilíři v~Praze půl hrdiny \\
u~mostu státi spatříte.
\end{verse}

\begin{verse}
Hlava zvětrala i spláchly ji deště \\
a prsa rozbil švédský boj; \\
než břich a nohy stojí potud ještě, \\
i pošetilé pýchy stroj.
\end{verse}

\begin{verse}
Nemluvte marně: „Starobylých věků \\
zpukřelé, vetché kamení!" \\
Vězte, že toto dnešních vašich reků \\
osudné jesti znamení!
\end{verse}

\begin{verse}
Slyšte a pilně važte moje slova: \\
S~nadějí nic se nenoste, \\
leč nad tím břichem vřelé srdce znova \\
a pravá hlava naroste!
\end{verse}