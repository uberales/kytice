\PoemTitle{Vodník}
\begin{verse}
I~\end{verse}

\begin{verse}
Na topole nad jezerem \\
seděl vodník podvečerem: \\
„Sviť, měsíčku, sviť, \\
ať mi šije niť.
\end{verse}

\begin{verse}
Šiju, šiju si botičky \\
do sucha i do vodičky: \\
sviť, měsíčku, sviť, \\
ať mi šije niť.
\end{verse}

\begin{verse}
Dnes je čtvrtek, zejtra pátek -- \\
šiju, šiju si kabátek: \\
sviť, měsíčku, sviť, \\
ať mi šije niť.
\end{verse}

\begin{verse}
Zelené šaty, botky rudé, \\
zejtra moje svatba bude: \\
sviť, měsíčku, sviť, \\
ať mi šije niť."
\end{verse}

\begin{verse}
II
\end{verse}

\begin{verse}
Ráno, raníčko panna vstala, \\
prádlo si v~uzel zavázala: \\
„Půjdu, matičko, k~jezeru, \\
šátečky sobě vyperu."
\end{verse}

\begin{verse}
„Ach nechoď, nechoď na jezero, \\
zůstaň dnes doma, moje dcero! \\
Já měla zlý té noci sen: \\
nechoď, dceruško, k~vodě ven.
\end{verse}

\begin{verse}
Perly jsem tobě vybírala, \\
bíle jsem tebe oblíkala, \\
v~sukničku jako z~vodních pěn: \\
nechoď, dceruško, k~vodě ven.
\end{verse}

\begin{verse}
Bílé šatičky smutek tají, \\
v~perlách se slzy ukrývají, \\
a pátek nešťastný je den, \\
nechoď, dceruško, k~vodě ven."--
\end{verse}

\begin{verse}
Nemá dceruška, nemá stání, \\
k~jezeru vždy ji cos pohání, \\
k~jezeru vždy ji cos nutí, \\
nic doma, nic jí po chuti. --
\end{verse}

\begin{verse}
První šáteček namočila -- \\
tu se s~ní lávka prolomila, \\
a po mladičké dívčině \\
zavířilo se v~hlubině.
\end{verse}

\begin{verse}
Vyvalily se vlny zdola, \\
roztáhnuly se v~šírá kola; \\
a na topole podle skal \\
zelený mužík zatleskal.
\end{verse}

\begin{verse}
III
\end{verse}

\begin{verse}
Nevesely truchlivy \\
jsou ty vodní kraje, \\
kde si v~trávě pod leknínem \\
rybka s~rybkou hraje.
\end{verse}

\begin{verse}
Tu slunéčko nezahřívá, \\
větřík nezavěje: \\
chladno, ticho -- \\
jako žel v~srdci bez naděje.
\end{verse}

\begin{verse}
Nevesely truchlivy \\
jsou ty kraje vodní; \\
v~poloutmě a v~polousvětle \\
mine tu den po dni.
\end{verse}

\begin{verse}
Dvůr vodníkův prostranný, \\
bohatství v~něm dosti; \\
však bezděky jen se v~něm \\
zastavují hosti.
\end{verse}

\begin{verse}
A~kdo jednou v~křišťálovou \\
bránu jeho vkročí, \\
sotva ho kdy uhlédají \\
jeho milých oči.
\end{verse}

\begin{verse}
Vodník sedí mezi vraty, \\
spravuje své sítě \\
a ženuška jeho mladá \\
chová malé dítě.
\end{verse}

\begin{verse}
„Hajej, dadej, mé děťátko, \\
můj bezděčný synu! \\
Ty se na mne usmíváš, \\
já žalostí hynu.
\end{verse}

\begin{verse}
Ty radostně vypínáš \\
ke mně ručky obě; \\
a já bych se radš viděla \\
tam na zemi v~hrobě.
\end{verse}

\begin{verse}
Tam na zemi za kostelem \\
u~černého kříže, \\
aby má matička zlatá \\
měla ke mně blíže.
\end{verse}

\begin{verse}
Hajej, dadej, synku můj, \\
můj malý vodníčku! \\
Kterak nemám vzpomínati \\
smutná na matičku?
\end{verse}

\begin{verse}
Starala se ubohá, \\
komu vdá mne, komu, \\
však ani se nenadálá, \\
vybyla mne z~domu!
\end{verse}

\begin{verse}
Vdala jsem se, vdala již, \\
ale byly chyby: \\
starosvati -- černí raci, \\
a družičky -- ryby!
\end{verse}

\begin{verse}
A~můj muž -- bůh polituj! \\
mokře chodí v~suše \\
a ve vodě pod hrnečky \\
střádá lidské duše.
\end{verse}

\begin{verse}
Hajej, dadej, můj synáčku \\
s~zelenými vlásky! \\
Nevdala se tvá matička \\
ve příbytek lásky.
\end{verse}

\begin{verse}
Obluzena, polapena \\
v~ošemetné sítě, \\
nemá žádné zde radosti \\
leč tebe, mé dítě!" --
\end{verse}

\begin{verse}
„Co to zpíváš, ženo má? \\
Nechci toho zpěvu! \\
Tvoje píseň proklatá \\
popouzí mne k~hněvu.
\end{verse}

\begin{verse}
Nic nezpívej, ženo má! \\
V~těle žluč mi kyne: \\
sic učiním rybou tebe \\
jako mnohé jiné!" --
\end{verse}

\begin{verse}
„Nehněvej se, nehněvej, \\
vodníku, můj muži! \\
Neměj za zlé rozdrcené, \\
zahozené růži.
\end{verse}

\begin{verse}
Mladosti mé jarý štěp \\
přelomil jsi v~půli; \\
a nic jsi mi po tu dobu \\
neučinil k~vůli.
\end{verse}

\begin{verse}
Stokrát jsem tě prosila, \\
přimlouvala sladce, \\
bys mi na čas, na kratičký, \\
dovolil k~mé matce.
\end{verse}

\begin{verse}
Stokrát jsem tě prosila \\
v~slzí toku mnohém, \\
bych jí ještě naposledy \\
mohla dáti sbohem!
\end{verse}

\begin{verse}
Stokrát jsem tě prosila, \\
na kolena klekla; \\
ale kůra srdce tvého \\
ničím neobměkla!
\end{verse}

\begin{verse}
Nehněvej se, nehněvej, \\
vodníku, můj pane, \\
anebo se rozhněvej, \\
co díš, ať se stane.
\end{verse}

\begin{verse}
A~chceš-li mne rybou míti, \\
abych byla němá, \\
učiň mne radš kamenem, \\
jenž paměti nemá.
\end{verse}

\begin{verse}
Učiň mne radš kamenem \\
bez mysli a citu, \\
by mi věčně žel nebylo \\
slunečního svitu!" --
\end{verse}

\begin{verse}
„Rád bych, ženo, rád bych já \\
věřil tvému slovu; \\
ale rybka v~šírém moři -- \\
kdo ji lapí znovu?
\end{verse}

\begin{verse}
Nezbraňoval bych ti já \\
k~matce tvojí chůze; \\
ale liché mysli ženské \\
obávám se tuze!
\end{verse}

\begin{verse}
Nuže -- dovolím ti já, \\
dovolím ti z~důli; \\
však poroučím, ať mi věrně \\
splníš moji vůli.
\end{verse}

\begin{verse}
Neobjímej matky své, \\
ani duše jiné; \\
sic pozemská tvoje láska \\
s~nezemskou se mine.
\end{verse}

\begin{verse}
Neobjímej nikoho \\
z~rána do večera; \\
před klekáním pak se zase \\
vrátíš do jezera.
\end{verse}

\begin{verse}
Od klekání do klekání \\
dávám lhůtu tobě; \\
avšak mi tu na jistotu \\
zůstavíš to robě."
\end{verse}

\begin{verse}
IV
\end{verse}

\begin{verse}
Jaké, jaké by to bylo \\
bez slunéčka podletí? \\
Jaké bylo by shledání \\
bez vroucího objetí? \\
A~když dcera v~dlouhém čase \\
matku svou obejme zase, \\
aj, kdo může za zlé míti \\
laskavému dítěti?
\end{verse}

\begin{verse}
Celý den se v~pláči těší \\
s~matkou žena z~jezera: \\
„Sbohem, má matičko zlatá! \\
Ach, bojím se večera!"-- \\
"Neboj se, má duše drahá, \\
nic se neboj toho vraha; \\
nedopustím, by tě v~moci měla \\
vodní příšera!" --
\end{verse}

\begin{verse}
Přišel večer. -- Muž zelený \\
chodí venku po dvoře; \\
dvéře klínem zastrčeny, \\
matka s~dcerou v~komoře. \\
„Neboj se, má drahá duše, \\
nic ti neuškodí v~suše, \\
vrah jezerní nemá k~tobě \\
žádné moci nahoře." --
\end{verse}

\begin{verse}
Když klekání odzvonili, \\
buch buch! venku na dvéře: \\
„Pojď již domů, ženo moje, \\
nemám ještě večeře." -- \\
"Vari od našeho prahu, \\
vari pryč, ty lstivý vrahu, \\
a co dřív jsi večeříval, \\
večeř zase v~jezeře!"--
\end{verse}

\begin{verse}
O~půlnoci buch buch! zase \\
na ty dvéře zpukřelé: \\
„Pojď již domů, ženo moje, \\
pojď mi ustlat postele." -- \\
„Vari od našeho prahu, \\
vari pryč, ty lstivý vrahu, \\
a kdo tobě prve stlával, \\
ať ti zase ustele!"--
\end{verse}

\begin{verse}
A~potřetí buch buch! zase, \\
když se šeřil ranní svit: \\
„Pojď již domů, ženo moje, \\
dítě pláče, dej mu pít!" \\
„Ach matičko, muka, muka -- \\
pro děťátko srdce puká! \\
Matko má, matičko zlatá, \\
nech mne, nech mne zase jít!" --
\end{verse}

\begin{verse}
„Nikam nechoď, dcero moje, \\
zradu kuje vodní vrah; \\
ač že péči máš o~dítě, \\
mně o~tebe větší strach. \\
Vari, vrahu, do jezera, \\
nikam nesmí moje dcera; \\
a pláče-li tvé děťátko, \\
přines je sem na náš práh."
\end{verse}

\begin{verse}
Na jezeře bouře hučí, \\
v~bouři dítě naříká; \\
nářek ostře bodá v~duši, \\
potom náhle zaniká. \\
„Ach matičko, běda, běda, \\
tím pláčem mi krev usedá; \\
matko má, matičko zlatá, \\
strachuji se vodníka!"--
\end{verse}

\begin{verse}
Něco padlo. -- Pode dveřmi \\
mok se jeví -- krvavý; \\
a když stará otevřela, \\
kdo leknutí vypraví! \\
Dvě věci tu v~krvi leží -- \\
mráz po těle hrůzou běží: \\
dětská hlava bez tělíčka \\
a tělíčko bez hlavy.
\end{verse}