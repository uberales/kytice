\PoemTitle{Dceřina kletba}

\begin{verse}
Což jsi se tak zasmušila, \\
dcero má \\
což jsi se tak zasmušila? \\
Veselá jsi jindy byla, \\
nyní přestal tobě smích!
\end{verse}

\begin{verse}
„Zabila jsem holoubátko, \\
matko má, \\
zabila jsem holoubátko -- \\
opuštěné jediňátko -- \\
bílé bylo jako sníh!"
\end{verse}

\begin{verse}
Holoubátko to nebylo, \\
dcero má, \\
holoubátko to nebylo -- \\
líčko se ti proměnilo \\
a potrhán je tvůj vzhled!
\end{verse}

\begin{verse}
„Och, zabila jsem děťátko, \\
matko má, \\
och, zabila jsem děťátko, \\
své ubohé zrozeňátko -- \\
žalostí bych pošla hned!"
\end{verse}

\begin{verse}
A~co míníš učiniti, \\
dcero má, \\
a co míníš učiniti, \\
kterak vinu napraviti \\
a smířiti boží hněv?
\end{verse}

\begin{verse}
„Půjdu hledat květu toho, \\
matko má, \\
půjdu hledat květu toho, \\
kterýž snímá viny mnoho \\
a vzbouřenou chladí krev."
\end{verse}

\begin{verse}
A~kde najdeš toho květu, \\
dcero má, \\
a kde najdeš toho květu \\
po všem široširém světu, \\
v~které roste zahrádce?
\end{verse}

\begin{verse}
„Tam za branou nad vršíkem, \\
matko má, \\
tam za branou nad vršíkem, \\
na tom sloupu se hřebíkem, \\
na konopné oprátce!"
\end{verse}

\begin{verse}
A~co vzkážeš hochu tomu, \\
dcero má, \\
a co vzkážeš hochu tomu, \\
jenž chodíval k~nám do domu \\
a s~tebou se těšíval?
\end{verse}

\begin{verse}
„Vzkazuji mu požehnání, \\
matko má, \\
vzkazuji mu požehnání -- \\
červa v~duši do skonání, \\
že mi zrádně mluvíval!"
\end{verse}

\begin{verse}
A~co necháš svojí matce, \\
dcero má, \\
a co necháš svojí matce, \\
jež tě milovala sladce \\
a draze tě chovala?
\end{verse}

\begin{verse}
„Kletbu zůstavuji tobě, \\
matko má, \\
kletbu zůstavuji tobě, \\
bys nenašla místa v~hrobě, \\
žes mi zvůli dávala!"
\end{verse}