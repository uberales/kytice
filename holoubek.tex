\PoemTitle{Holoubek}

\begin{verse}
Okolo hřbitova \\
cesta úvozová; \\
šla tudy, plakala \\
mladá, hezká vdova.
\end{verse}

\begin{verse}
Plakala, želela \\
pro svého manžela: \\
neb tudy naposled \\
jej doprovázela. --
\end{verse}

\begin{verse}
Od bílého dvora \\
po zelené louce \\
jede pěkný panic, \\
péro na klobouce.
\end{verse}

\begin{verse}
„Neplač, nenaříkej, \\
mladá, hezká vdovo, \\
škoda by tvých očí, \\
slyš rozumné slovo.
\end{verse}

\begin{verse}
Neplač, nenaříkej, \\
vdovo, pěkná růže, \\
a když muž ti umřel, \\
vezmi mne za muže.“ --
\end{verse}

\begin{verse}
Jeden den plakala, \\
druhý ticho minul, \\
třetího žel její \\
pomalu zahynul.
\end{verse}

\begin{verse}
V~témž dni umrlého \\
z~mysli vypustila: \\
než měsíc uplynul, \\
k~svatbě šaty šila.
\end{verse}

\begin{verse}
Okolo hřbitova \\
veselejší cesta: \\
jedou tudy, jedou \\
ženich a nevěsta.
\end{verse}

\begin{verse}
Byla svatba, byla \\
hlučná a veselá: \\
nevěsta v~objetí \\
nového manžela.
\end{verse}

\begin{verse}
Byla svatba, byla, \\
hudba pěkně hrála: \\
on ji k~sobě vinul, \\
ona jen se smála.
\end{verse}

\begin{verse}
Směj se, směj, nevěsto, \\
pěkně ti to sluší: \\
nebožtík pod zemí, \\
ten má hluché uši!
\end{verse}

\begin{verse}
Objímej milého, \\
netřeba se báti: \\
rakev dosti těsná -- \\
ten se neobrátí! \\
Líbej si je, líbej, \\
ty žádané líce: \\
komus namíchala, \\
neobživne více! --
\end{verse}

\begin{verse}
*
\end{verse}

\begin{verse}
Běží časy, běží, \\
všecko s~sebou mění: \\
co nebylo, přijde \\
co bývalo, není.
\end{verse}

\begin{verse}
Běží časy, běží, \\
rok jako hodina: \\
jedno však nemizí: \\
pevně stojí vina.
\end{verse}

\begin{verse}
Tři roky minuly, \\
co nebožtík leží; \\
na jeho pahorku \\
tráva roste svěží.
\end{verse}

\begin{verse}
Na pahorku tráva, \\
u~hlavy mu doubek, \\
na doubku sedává \\
běloučký holoubek.
\end{verse}

\begin{verse}
Sedává, sedává, \\
přežalostně vrká: \\
každý, kdo uslyší, \\
srdce jemu puká.
\end{verse}

\begin{verse}
Nepuká tak jiným \\
jako jedné ženě: \\
z~hlavy si rve vlasy, \\
volá uděšeně:
\end{verse}

\begin{verse}
„Nehoukej, nevolej, \\
nehuč mi tak v~uši: \\
tvá píseň ukrutná \\
probodá mi duši!
\end{verse}

\begin{verse}
Nehoukej, nežaluj, \\
hlava se mi točí; \\
aneb mi zahoukej, \\
a se mi rozskočí!“ --
\end{verse}

\begin{verse}
Teče voda, teče, \\
vlna vlnu stíhá \\
a mezi vlnami \\
bílý šat se míhá.
\end{verse}

\begin{verse}
Tu vyplývá noha, \\
tam zas ruka bledá: \\
žena nešťastnice \\
hrobu sobě hledá!
\end{verse}

\begin{verse}
Vytáhli ji na břeh, \\
zahrabali skrytě, \\
kde cesty pěšiny \\
křižují se v~žitě.
\end{verse}

\begin{verse}
Nižádného hrobu \\
jí býti nemělo: \\
jen kámen veliký \\
tlačí její tělo.
\end{verse}

\begin{verse}
Však nelze kamenu \\
tak těžko ležeti, \\
jako jí na jménu \\
spočívá prokletí!
\end{verse}

