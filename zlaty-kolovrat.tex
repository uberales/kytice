\PoemTitle{Zlatý kolovrat}

\begin{verse}
I~\end{verse}

\begin{verse}
Okolo lesa pole lán, \\
hoj jede, jede z~lesa pán, \\
na vraném bujném jede koni, \\
vesele podkovičky zvoní, \\
jede sám a sám.
\end{verse}

\begin{verse}
A~před chalupou s~koně hop \\
a na chalupu: klop, klop, klop! \\
„Hola hej ! otevřte mi dvéře, \\
zbloudil jsem při lovení zvěře, \\
dejte vody pít!“
\end{verse}

\begin{verse}
Vyšla dívčina jako květ, \\
neviděl také krásy svět; \\
přinesla vody ze studnice, \\
stydlivě sedla u~přeslice, \\
předla, předla len.
\end{verse}

\begin{verse}
Pán stojí, nevěda, co chtěl, \\
své velké žízně zapomněl; \\
diví se tenké, rovné niti, \\
nemůže očí odvrátiti \\
z~pěkné přadleny.
\end{verse}

\begin{verse}
„Svobodna-li jest ruka tvá, \\
ty musíš býti žena má!“ \\
dívčinu k~boku svému vine -- \\
„Ach pane, nemám vůle jiné, \\
než jak máti chce.“
\end{verse}

\begin{verse}
„A kde je, děvče, máti tvá? \\
Nikohoť nevidím tu já.“ -- \\
„Ach pane, má nevlastní máti \\
zejtra se s~dcerou domů vrátí, \\
vyšly do města.“
\end{verse}

\begin{verse}
II
\end{verse}

\begin{verse}
Okolo lesa pole lán, \\
hoj jede, jede zase pán; \\
na vraném bujném jede koni, \\
vesele podkovičky zvoní, \\
přímo k~chaloupce.
\end{verse}

\begin{verse}
A~před chalupou s~koně hop \\
a na chalupu: klop, klop, klop! \\
„Hola! otevřte, milí lidi, \\
ať oči moje brzo vidí \\
potěšení mé!“
\end{verse}

\begin{verse}
Vyšla babice, kůže a kost: \\
„Hoj, co nám nese vzácný host?“ \\
„Nesu ti, nesu v~domě změnu, \\
chci tvoji dceru za svou ženu, \\
tu tvou nevlastní.“
\end{verse}

\begin{verse}
„Hoho, panáčku, div a div, \\
kdo by pomyslil jaktěživ? \\
Pěkně vás vítám, vzácný hoste \\
však ani nevím, pane, kdo jste? \\
Jak jste přišel k~nám?“
\end{verse}

\begin{verse}
„Jsem této země král a pán, \\
náhodou včera zavolán: \\
dám tobě stříbro, dám ti zlato, \\
dej ty mně svoji dceru za to, \\
pěknou přadlenu.“
\end{verse}

\begin{verse}
„Ach pane králi, div a div, \\
kdo by se nadál jaktěživ? \\
Vždyť nejsme hodny, pane králi, \\
kéž bychom záslužněji stály \\
v~milosti vaší!
\end{verse}

\begin{verse}
Ale však radu, radu mám: \\
za cizí -- dceru vlastní dám; \\
jeť podobna té druhé právě \\
jak oko oku v~jedné hlavě -- \\
její nit -- hedbáv!“
\end{verse}

\begin{verse}
„Špatná je, babo, rada tvá, \\
vykonej, co poroučím já: \\
zejtra, a den se ráno zjasní, \\
provodíš dceru svou nevlastní \\
na královský hrad!“
\end{verse}

\begin{verse}
III
\end{verse}

\begin{verse}
„Vstávej, dceruško, již je čas, \\
pan král již čeká, bude kvas; \\
však jsem já ani netušila -- \\
nu bodejž dobře pořídila \\
v~královském hradě!“
\end{verse}

\begin{verse}
„Stroj se, sestřičko moje, stroj, \\
v~královském hradě bude hoj; \\
vysoko jsi se podívala, \\
nízko mne, hleďte, zanechala -- \\
nu jen zdráva buď!“
\end{verse}

\begin{verse}
„Pojď již, Dorničko naše, pojď, \\
aby se nehněval tvůj choť; \\
až budeš v~lese na rozhraní, \\
na domov nevzpomeneš ani -- \\
pojď jen honem, pojď!“
\end{verse}

\begin{verse}
„Matko, matičko, řekněte \\
nač s~sebou ten nůž béřete?“ -- \\
„Nůž bude dobrý -- někde v~chladu \\
vypíchnem oči zlému hadu -- \\
pojď jen honem, pojď!“
\end{verse}

\begin{verse}
„Sestro, sestřičko, řekněte, \\
nač tu sekeru nesete?“ \\
„Sekera dobrá -- někde v~keři \\
useknem hnáty líté zvěři -- \\
pojď jen honem, pojď!“
\end{verse}

\begin{verse}
A~když již přišly v~chlad a keř: \\
„Hoj, ty jsi ten had, tys ta zvěř!“ \\
Hory a doly zaplakaly, \\
kterak dvě ženy nakládaly \\
s~pannou ubohou!
\end{verse}

\begin{verse}
„Nyní se s~panem králem těš, \\
těš se s~ním, kterakkoli chceš: \\
objímej jeho svěží tělo, \\
pohlížej na to jasné čelo, \\
pěkná přadleno!“
\end{verse}

\begin{verse}
„Mamičko, kterak udělám, \\
kam oči a ty hnáty dám?“ -- \\
„Nenechávej jich podle těla, \\
ať někdo jich zas nepřidělá -- \\
radš je s~sebou vem.“
\end{verse}

\begin{verse}
A~když již zašly za tu chvoj: \\
„Nic ty se, dcero má, neboj, \\
však jsi podobna té tam právě \\
jak oko oku v~jedné hlavě -- \\
neboj ty se nic!“
\end{verse}

\begin{verse}
A~když již byly hradu blíž, \\
pan král vyhlíží z~okna již; \\
vychází s~pány svými v~cestu, \\
přivítá matku i nevěstu, \\
zrady netuše.
\end{verse}

\begin{verse}
I~byla svatba -- zralý hřích, \\
panna nevěsta samý smích; \\
i byly hody, radování, \\
plesy a hudby bez ustání \\
do sedmého dne.
\end{verse}

\begin{verse}
A~když zasvítal osmý den, \\
král musí jíti s~vojsky ven: \\
„Měj se tu dobře, paní moje, \\
já jedu do krutého boje, \\
na nepřítele.
\end{verse}

\begin{verse}
Navrátím-li se z~bitvy zpět, \\
omladne naší lásky květ! \\
Zatím na věrnou mou památku \\
hleď sobě pilně kolovrátku, \\
pilně doma přeď!“
\end{verse}

\begin{verse}
IV
\end{verse}

\begin{verse}
V~hluboké pusté křovině \\
jak se tam vedlo dívčině? \\
Šest otevřených proudů bylo, \\
z~nich se jí živobytí lilo \\
na zelený mech.
\end{verse}

\begin{verse}
Vzešlo jí náhle štěstí moc, \\
nynčko jí hrozí smrti noc: \\
tělo již chladne, krev se sedá -- \\
běda té době, běda, běda, \\
když ji spatřil král!
\end{verse}

\begin{verse}
A~tu se z~lesních kdesi skal \\
stařeček nevídaný vzal: \\
šedivé vousy po kolena -- \\
to tělo vloživ na ramena \\
v~jeskyni je nes.
\end{verse}

\begin{verse}
„Vstaň, mé pachole, běž, je chvat, \\
vezmi ten zlatý kolovrat: \\
v~královském hradě jej prodávej, \\
za nic jiného však nedávej \\
nežli za nohy.“ --
\end{verse}

\begin{verse}
Pachole v~bráně sedělo, \\
zlatý kolovrat drželo. \\
Královna z~okna vyhlížela: \\
„Kéž bych ten kolovrátek měla \\
z~ryzího zlata!“
\end{verse}

\begin{verse}
„Jděte se, matko, pozeptat, \\
zač je ten zlatý kolovrat?“ \\
„Kupte, paničko, drahý není, \\
můj otec příliš nevycení: \\
za dvě nohy jest.“
\end{verse}

\begin{verse}
„Za nohy? Ajaj, divná věc! \\
Ale já chci jej míti přec: \\
jděte, mamičko, do komory, \\
jsou tam ty nohy naší Dory, \\
dejte mu je zaň.“
\end{verse}

\begin{verse}
Pachole nohy přijalo, \\
do lesa zpátky spěchalo. -- \\
„Podej mi, chlapče, živé vody, \\
nechť bude tělo beze škody, \\
jako bývalo.“
\end{verse}

\begin{verse}
A~ránu k~ráně přiložil, \\
a v~nohou oheň zas ožil, \\
a v~jeden celek srostlo tělo, \\
jako by vždycky bylo celo, \\
bez porušení.
\end{verse}

\begin{verse}
„Jdi, mé pachole, k~polici, \\
vezmi tu zlatou přeslici: \\
v~královském hradě ji prodávej, \\
za nic jiného vak nedávej, \\
nežli za ruce.“ --
\end{verse}

\begin{verse}
Pachole v~bráně sedělo, \\
přeslici v~rukou drželo. \\
Královna z~okna vyhlížela: \\
„Och, kéž bych tu přesličku měla \\
ke kolovrátku!
\end{verse}

\begin{verse}
Vstaňte, mamičko, z~lavice, \\
ptejte se, zač ta přeslice?“ -- \\
„Kupte, paničko, drahá není, \\
můj otec příliš nevycení, \\
za dvě ruce jest.“
\end{verse}

\begin{verse}
„Za ruce!? Divná, divná věc! \\
Ale já ji chci míti přec: \\
jděte, mamičko, do komory, \\
jsou tam ty ruce naší Dory, \\
přineste mu je.“
\end{verse}

\begin{verse}
Pachole ruce přijalo, \\
do lesa zpátky spěchalo. -- \\
„Podej mi, chlapče, živé vody, \\
nechť bude tělo beze škody, \\
jako bývalo.“
\end{verse}

\begin{verse}
A~ránu k~ráně přiložil, \\
a v~rukou oheň zas ožil; \\
a v~jeden celek srostlo tělo, \\
jako by vždycky bylo celo, \\
bez porušení.
\end{verse}

\begin{verse}
„Skoč, hochu, na cestu se měj, \\
mám zlatý kužel na prodej: \\
v~královském hradě jej prodávej, \\
za nic jiného však nedávej, \\
nežli za oči.“ --
\end{verse}

\begin{verse}
Pachole v~bráně sedělo, \\
zlatý kuželík drželo. \\
Královna z~okna vyhlížela: \\
„Kéž bych ten kuželíček měla \\
na tu přesličku!
\end{verse}

\begin{verse}
Vstaňte, mamičko, jděte zas, \\
ptejte se, zač ten kužel as?“ -- \\
„Za oči, paní, jinak není, \\
tak mi dal otec poručení, \\
za dvě oči jest.“
\end{verse}

\begin{verse}
„Za oči!? Neslýchaná věc! \\
A~kdo je, chlapče, tvůj otec?“ -- \\
„Netřeba znáti otce mého: \\
kdo by ho hledal, nenajde ho, \\
jinak přijde sám.“ --
\end{verse}

\begin{verse}
„Mámo, mamičko, co počít? \\
A~já ten kužel musím mít! \\
Jděte tam zase do komory, \\
jsou tam ty oči naší Dory, \\
ať je odnese.“
\end{verse}

\begin{verse}
Pachole oči přijalo, \\
do lesa zpátky spěchalo. -- \\
„Podej mi, chlapče, živé vody, \\
nechť bude tělo beze škody, \\
jako kdy prve.“
\end{verse}

\begin{verse}
A~oči v~důlky položil, \\
a zhaslý oheň zas ožil; \\
a panna vůkol pohlížela -- \\
však nikoho tu neviděla, \\
než se samotnu.
\end{verse}

\begin{verse}
V~\end{verse}

\begin{verse}
A~když byly tři neděle, \\
král jede z~vojny vesele: \\
„A jak se má, má paní milá, \\
a zdalis pamětliva byla \\
mých posledních slov?“
\end{verse}

\begin{verse}
„Och, já je v~srdci nosila, \\
a hleďte, co jsem koupila: \\
jediný mezi kolovraty, \\
přeslici, kužel -- celý zlatý, \\
vše to z~lásky k~vám!“
\end{verse}

\begin{verse}
„Pojď se, má paní, posadit, \\
upřeď mi z~lásky zlatou nit.“ -- \\
Ke kolovrátku chutě sedla, \\
jak zatočila, celá zbledla -- \\
běda, jaký zpěv!
\end{verse}

\begin{verse}
„Vrrr -- zlou to předeš nit! \\
Přišla jsi krále ošidit: \\
nevlastní sestru jsi zabila, \\
údův a očí ji zbavila -- \\
vrrr -- zlá to nit!“
\end{verse}

\begin{verse}
„Jaký to kolovrátek máš? \\
A~jak mi divně na něj hráš! \\
Zahraj mi, paní, ještě znova, \\
nevím, co chtějí tato slova: \\
přeď, má paní, přeď!“
\end{verse}

\begin{verse}
„Vrrr -- zlou to předeš nit! \\
Chtěla jsi krále ošidit: \\
pravou nevěstu jsi zabila \\
a sama ses jí učinila -- \\
vrrr -- zlá to nit!“
\end{verse}

\begin{verse}
„Hó, strašlivě mi, paní, hráš! \\
Nejsi tak, jak se býti zdáš! \\
Zahraj mi, paní, do třetice, \\
abych uslyšel ještě více: \\
přeď, má paní, přeď!“
\end{verse}

\begin{verse}
„Vrrr -- zlou to předeš nit! \\
Přišla jsi krále ošidit: \\
sestra tvá v~lese, v~duté skále, \\
ukradla jsi jí chotě krále -- \\
vrrr -- zlá to nit!“
\end{verse}

\begin{verse}
Jak ta slova král uslyšel, \\
skočil na vrance, k~lesu jel; \\
hledal a volal v~šíré lesy: \\
„Kdes, má Dorničko, kde jsi, kde jsi, \\
kdes, má rozmilá?“
\end{verse}

\begin{verse}
VI
\end{verse}

\begin{verse}
Od lesa k~hradu polí lán, \\
hoj jede, jede s~paní pán; \\
na vraném bujném jedou koni, \\
vesele podkovičky zvoní, \\
na královský hrad.
\end{verse}

\begin{verse}
I~přišla svatba zase zpět, \\
panna nevěsta jako květ; \\
i byly hody, radování, \\
hudby a plesy bez ustání \\
po tři neděle.
\end{verse}

\begin{verse}
A~co ta matka babice? \\
A~co ta dcera hadice? -- \\
Hoj, vyjí čtyři vlci v~lese, \\
každý po jedné noze nese \\
ze dvou ženských těl.
\end{verse}

\begin{verse}
Z~hlavy jim oči vyňaty, \\
ruce i nohy uťaty: \\
co prve panně udělaly, \\
toho teď na se dočekaly \\
v~lese hlubokém.
\end{verse}

\begin{verse}
A~co ten zlatý kolovrat? \\
Jakou teď píseň bude hrát? \\
Jen do třetice zahrát přišel, \\
pak ho již nikdo neuslyšel, \\
ani nespatřil.
\end{verse}