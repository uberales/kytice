\PoemTitle{Štědrý den}

\begin{verse}
I~\end{verse}

\begin{verse}
Tma jako v~hrobě, mráz v~okna duje, \\
v~světnici teplo u~kamen; \\
v~krbu se svítí, stará podřimuje, \\
děvčata předou měkký len.
\end{verse}

\begin{verse}
„Toč se a vrč, můj kolovrátku, \\
ejhle adventu již nakrátku \\
a blízko, blizoučko Štědrý den!
\end{verse}

\begin{verse}
Mílotě děvčeti přísti, mílo \\
za smutných zimních večerů; \\
neb nebude darmo její dílo, \\
tu pevnou chová důvěru.
\end{verse}

\begin{verse}
I~přijde mládenec za pilnou pannou, \\
řekne: Pojď za mne, dívko má, \\
budiž ty mi ženkou milovanou, \\
věrným ti mužem budu já.
\end{verse}

\begin{verse}
Já tobě mužem, ty mně ženkou, \\
dej ruku, děvče rozmilé! -- \\
A~dívka, co předla přízi tenkou, \\
svatební šije košile.
\end{verse}

\begin{verse}
Toč se a vrč, můj kolovrátku, \\
však jest adventu již nakrátku \\
a přede dveřmi Štědrý den!“
\end{verse}

\begin{verse}
II
\end{verse}

\begin{verse}
Hoj, ty Štědrý večere, \\
ty tajemný svátku, \\
cože komu dobrého \\
neseš na památku?
\end{verse}

\begin{verse}
Hospodáři štědrovku, \\
kravám po výslužce; \\
kohoutovi česneku, \\
hrachu jeho družce.
\end{verse}

\begin{verse}
Ovocnému stromoví \\
od večeře kosti \\
a zlatoušky na stěnu \\
tomu, kdo se postí.
\end{verse}

\begin{verse}
Hoj, já mladá dívčina, \\
srdce nezadané: \\
mně na mysli jiného, \\
jiného cos tane.
\end{verse}

\begin{verse}
Pod lesem, ach pod lesem, \\
na tom panském stavě, \\
stojí vrby stařeny, \\
sníh na šedé hlavě.
\end{verse}

\begin{verse}
Jedna vrba hrbatá \\
tajně dolů kývá, \\
kde se modré jezero \\
pod ledem ukrývá.
\end{verse}

\begin{verse}
Tu prý dívce v~půlnoci, \\
při luně pochodni, \\
souzený se zjeví \\
hoch ve hladině vodní.
\end{verse}

\begin{verse}
Hoj, mne půlnoc neleká, \\
ani liché vědy: \\
půjdu, vezmu sekeru, \\
prosekám ty ledy.
\end{verse}

\begin{verse}
I~nahlédnu v~jezero \\
hluboko -- hluboko, \\
milému se podívám \\
pevně okem v~oko.
\end{verse}

\begin{verse}
III
\end{verse}

\begin{verse}
Marie, Hana, dvě jména milá, \\
panny jak jarní růže květ: \\
která by z~obou milejší byla, \\
nikdo nemůže rozumět.
\end{verse}

\begin{verse}
Jestliže jedna promluví k~hochu, \\
do ohně by jí k~vůli šel; \\
pakli se druhá usměje trochu -- \\
na první zas by zapomněl! --
\end{verse}

\begin{verse}
Nastala půlnoc. Po nebi šíře \\
sbor vysypal se hvězdiček \\
jako ovečky okolo pastýře, \\
a pastýř jasný měsíček.
\end{verse}

\begin{verse}
Nastala půlnoc, všech nocí máti, \\
půlnoc po Štědrém večeru; \\
na mladém sněhu svěží stopu znáti \\
ode vsi přímo k~jezeru.
\end{verse}

\begin{verse}
Ta jedna klečí, nad vodou líčko; \\
ta druhá stojí podle ní: \\
„Hano, Haničko, zlaté srdíčko, \\
jaké tam vidíš vidění?“
\end{verse}

\begin{verse}
„Ach, vidím domek -- ale jen v~šeře -- \\
jako co Václav ostává -- \\
však již se jasní -- ach, vidím dvéře, \\
ve dveřích mužská postava!
\end{verse}

\begin{verse}
Na těle kabát zeleni temné, \\
klobouk na stranu -- znám jej, znám! \\
Na něm ta kytka, co dostal ode mne -- \\
můj milý bože! Václav sám!!“
\end{verse}

\begin{verse}
Na nohy skočí, srdce jí bije, \\
druhá přikleká vedle ní: \\
„Zdař bůh, má milá, zlatá Marie, \\
jaké ty vidíš vidění?“
\end{verse}

\begin{verse}
„Ach, vidím, vidím -- je mlhy mnoho, \\
všecko je mlhou zatmělé; \\
červená světla blýskají z~toho -- \\
zdá mi se býti v~kostele.
\end{verse}

\begin{verse}
Něco se černá mezi bílými -- \\
však mi se rozednívá již: -- \\
jsou to družičky, a mezi nimi proboha! \\
rakev -- černý kříž!“
\end{verse}

\begin{verse}
IV
\end{verse}

\begin{verse}
Vlažný větřík laškuje \\
po osení mladém; \\
sad i pole květovým \\
přioděny vnadem; \\
zazněla hudba od kostela zrána \\
a za ní hejsa! kvítím osypána \\
jede svatba řadem.
\end{verse}

\begin{verse}
Švárný ženich jako květ \\
v~kole svatebčanů, \\
kabát tmavě zelený, \\
klobouk v~jednu stranu: \\
tak viděla jej v~osudné té době, \\
tak si ji nyní domů vede k~sobě, \\
švárnou ženku Hanu.
\end{verse}

\begin{verse}
*
\end{verse}

\begin{verse}
Zašlo léto. Přes pole \\
chladné větry vějí. \\
Zvoní hrana. Na marách \\
tělo vynášejí: \\
bílé družičky, planoucí svíce; \\
pláč, bědování, trouby hlaholíce \\
z~hlubokosti znějí: \\
Miserere mei!
\end{verse}

\begin{verse}
Koho věnec zelený, \\
koho v~rakvi kryje? \\
Umřela, ach umřela \\
panenská lilie! \\
Vykvětla, jak by zalívána rosou, \\
uvadla, jak by podsečena kosou -- \\
ubohá Marie!
\end{verse}

\begin{verse}
V~\end{verse}

\begin{verse}
Nastala zima, mráz v~okna duje, \\
v~světnici teplo u~kamen; \\
v~krbu se svítí, stará polehuje, \\
děvčata zase předou len.
\end{verse}

\begin{verse}
„Toč se a vrč, můj kolovrátku, \\
však jest adventu zase nakrátku \\
a nedaleko Štědrý den!
\end{verse}

\begin{verse}
Ach ty Štědrý večere \\
noci divoplodné, \\
když si na tě vzpomenu, \\
k~srdci mne to bodne!
\end{verse}

\begin{verse}
Seděly jsme také tak \\
loni pohromadě: \\
a než rok se obrátil, \\
dvě nám chybí v~řadě!
\end{verse}

\begin{verse}
Jedna, hlavu zavitou, \\
košiličky šije; \\
druhá již tři měsíce \\
v~černé zemi hnije, \\
ubohá Marie!
\end{verse}

\begin{verse}
Seděly jsme také tak, \\
jako dnes a včera: \\
a než rok se obrátí -- \\
kde z~nás bude která?
\end{verse}

\begin{verse}
Toč se a vrč, můj kolovrátku, \\
všecko ve světě jen na obrátku \\
a život lidský jako sen!
\end{verse}

\begin{verse}
Však lépe v~mylné naději sníti, \\
před sebou čirou temnotu, \\
nežli budoucnost odhaliti, \\
strašlivou poznati jistotu!“\\
\end{verse}
