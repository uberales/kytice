\PoemTitle{Vrba}

\begin{verse}
Ráno sedá ke snídani, \\
táže se své mladé paní:
\end{verse}

\begin{verse}
„Paní moje, paní milá, \\
vždycky upřímná jsi byla,
\end{verse}

\begin{verse}
vždycky upřímná jsi byla -- \\
jednohos mi nesvěřila.
\end{verse}

\begin{verse}
Dvě léta jsme spolu nyní -- \\
jedno nepokoj mi činí.
\end{verse}

\begin{verse}
Paní moje, milá paní, \\
jaké je to tvoje spaní?
\end{verse}

\begin{verse}
Večer lehneš zdráva, svěží, \\
v~noci tělo mrtvo leží.
\end{verse}

\begin{verse}
Ani ruchu, ani sluchu, \\
ani zdání o~tvém duchu.
\end{verse}

\begin{verse}
Studené jest to tvé tělo, \\
jak by zpráchnivěti chtělo.
\end{verse}

\begin{verse}
Aniž to maličké dítě, \\
hořce plačíc, probudí tě. --
\end{verse}

\begin{verse}
Paní moje, paní zlatá, \\
zdali nemocí jsi jata?
\end{verse}

\begin{verse}
Jestli nemoc ta závada \\
nech, ať přijde moudrá rada.
\end{verse}

\begin{verse}
V~poli mnoho bylin stojí, \\
snad některá tebe zhojí.
\end{verse}

\begin{verse}
Pakli v~býlí není síly, \\
mocné slovo neomýlí.
\end{verse}

\begin{verse}
Mocné slovo mračna, \\
vodí v~bouři líté chrání lodi.
\end{verse}

\begin{verse}
Mocné slovo ohni káže, \\
skálu zdrtí, draka sváže.
\end{verse}

\begin{verse}
Jasnou hvězdu strhne z~nebe: \\
slovo mocné zhojí tebe." --
\end{verse}

\begin{verse}
„Ó pane můj, milý pane, \\
nechtěj dbáti řeči plané.
\end{verse}

\begin{verse}
Co souzeno při zrození, \\
tomu nikdež léku není.
\end{verse}

\begin{verse}
Co sudice komu káže, \\
slovo lidské nerozváže!
\end{verse}

\begin{verse}
Ač bezduchá na svém loži, \\
vždy jsem přece v~moci boží.
\end{verse}

\begin{verse}
Vždy jsem přece v~boží moci, \\
jenž mne chrání každé noci.
\end{verse}

\begin{verse}
Ač co mrtvé mi je spáti, \\
ráno duch se zase vrátí.
\end{verse}

\begin{verse}
Ráno zdráva vstáti mohu: \\
protož poruč pánu bohu!" --
\end{verse}

\begin{verse}
Darmo, paní, jsou tvá slova, \\
pán úmysl jiný chová.
\end{verse}

\begin{verse}
Sedí babka při ohnisku, \\
měří vodu z~misky v~misku,
\end{verse}

\begin{verse}
dvanáct misek v~jedné řadě. \\
Pán u~baby na poradě.
\end{verse}

\begin{verse}
„Slyšíš, matko, ty Víš mnoho: \\
víš, co potkati má koho,
\end{verse}

\begin{verse}
víš, kde se čí nemoc rodí, \\
kudy smrtná žena chodí.
\end{verse}

\begin{verse}
Pověz ty mi zjevně nyní, \\
co se s~mojí paní činí?
\end{verse}

\begin{verse}
Večer lehne zdráva, svěží, \\
v~noci tělo mrtvo leží,
\end{verse}

\begin{verse}
ani ruchu, ani sluchu, \\
ni zdání o~jejím duchu;
\end{verse}

\begin{verse}
studené jest její tělo, \\
jak by zpráchnivěti chtělo." --
\end{verse}

\begin{verse}
Kterak nemá mrtva býti, \\
když má jen půl živobytí?
\end{verse}

\begin{verse}
Ve dne s~tebou živa v~domě, \\
v~noci duše její v~stromě.
\end{verse}

\begin{verse}
Jdi k~potoku pod oborou, \\
najdeš vrbu s~bílou korou;
\end{verse}

\begin{verse}
žluté proutí roste na ní: \\
s~tou je duše tvojí paní!" --
\end{verse}

\begin{verse}
„Nechtěl jsem já paní míti, \\
aby s~vrbou měla žíti;
\end{verse}

\begin{verse}
paní má ať se mnou žije \\
a vrba ať v~zemi hnije." --
\end{verse}

\begin{verse}
Vzal sekeru na ramena, \\
uťal vrbu od kořena;
\end{verse}

\begin{verse}
padla těžce do potoka, \\
zašuměla od hluboka,
\end{verse}

\begin{verse}
zašuměla, zavzdychala, \\
jak by matka skonávala,
\end{verse}

\begin{verse}
jak by matka umírajíc, \\
po dítku se ohlédajíc. --
\end{verse}

\begin{verse}
„Jaký shon to k~mému domu? \\
Komu zní hodinka, komu?" --
\end{verse}

\begin{verse}
„Umřela tvá paní milá, \\
jak by kosou sťata byla;
\end{verse}

\begin{verse}
zdráva chodíc při své práci, \\
padla, jako strom se skácí;
\end{verse}

\begin{verse}
zavzdychala umírajíc, \\
po dítku se ohlédajíc." --
\end{verse}

\begin{verse}
„Ó běda mi, běda, běda, \\
paní zabil jsem nevěda,
\end{verse}

\begin{verse}
a z~děťátka v~tuž hodinu \\
učinil jsem sirotinu!
\end{verse}

\begin{verse}
Ó ty vrbo, vrbo bílá, \\
což jsi ty mne zarmoutila!
\end{verse}

\begin{verse}
Vzalas mi půl živobytí: \\
co mám s~tebou učiniti?" --
\end{verse}

\begin{verse}
„Dej mne z~vody vytáhnouti, \\
osekej mé žluté proutí;
\end{verse}

\begin{verse}
dej prkének nařezati, \\
kolébku z~nich udělati;
\end{verse}

\begin{verse}
na kolébku vlož děťátko, \\
ať nepláče ubožátko.
\end{verse}

\begin{verse}
Když se bude kolébati, \\
matka bude je chovati.
\end{verse}

\begin{verse}
Proutí zasaď podle vody, \\
by nevzalo žádné škody
\end{verse}

\begin{verse}
Až doroste hoch maličký, \\
bude řezat píšťaličky;
\end{verse}

\begin{verse}
na píšťalku bude pěti -- \\
se svou matkou rozprávěti!"
\end{verse}