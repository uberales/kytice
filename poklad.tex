\PoemTitle{Poklad}

\begin{verse}
I~\end{verse}

\begin{verse}
Na pahorku mezi buky \\
kostelíček s~věží nízkou; \\
z~věže pak slyšeti zvuky \\
hájem a sousední vískou. \\
Není zvuk to zvonka jemný, \\
tratící se v~blízké stráně: \\
dřevatě to rachot temný, \\
zvoucí lid do chrámu Páně. \\
A~tu z~vísky k~boží slávě \\
vzhůru běží zástup hojný: \\
veský lid to bohabojný, \\
a dnes Velký pátek právě. \\
V~chrámě truchlo: holé stěny; \\
oltář černá rouška kryje, \\
na roušce kříž upevněný; \\
v~kůru zpívají pašije. \\
A~hle, co se bělá v~lese, \\
v~černém lese za potokem
\end{verse}

\begin{verse}
Nějaká to veská žena, \\
ana v~náručí cos nese. \\
I~jde rychlým žena krokem, \\
svátečně jsouc oblečena, \\
tam tou strání za potokem -- \\
pacholátko malé nese.
\end{verse}

\begin{verse}
Běží žena, dolů běží, \\
pospíchá do chrámu Páně: \\
tu nablízku lesní stráně \\
kostel na pahorku leží. \\
A~v~úvale ku potoku \\
náhle ubystřuje kroků; \\
neb jak větřík volně věje, \\
z~kostela slyšeti pění: \\
v~kůru tam se právě pěje \\
Krista Pána umučení. \\
Běží, běží podle skály: \\
„Co to? Mám-li věřit oku? \\
Což mě moje smysly šálí?“ \\
Stane, ohlíží se kolem -- \\
rychle kroky zpět obrací, \\
stane zase, zas se vrací -- \\
„Tam ten les, a zde ty klesty, \\
tamto vede cesta polem -- \\
vždyť jsem nezbloudila z~cesty! \\
Bože, co se se mnou děje! \\
Což zde nejsem u~kamena? \\
Jaká se tu stala změna!“ \\
Zase stojí, zase spěje, \\
celá jsouci udivena, \\
oči rukou si protírá, \\
o~krok blíže se ubírá: \\
„Bože, jaká to tu změna!"
\end{verse}

\begin{verse}
Tu, kde z~divokého klestu, \\
od kostela tři sta kroků, \\
veliký čněl kámen v~cestu, \\
co se nyní jeví oku? \\
Jeví se tu ženě, jeví \\
vchodem vršek otevřený -- \\
vysvětliti sobě neví -- \\
kámen v~cestu postavený, \\
postavena celá skála \\
jak by od věků zde stála. \\
Jeví se tu, jeví ženě \\
chodba pod zemí, co síně \\
vyklenutá ve křemeně; \\
a tam, klenba kde se tratí, \\
ve tmavém pahorku klíně, \\
jakýs plamen znamenati. \\
I~hoří to jasnoběle, \\
jako v~noci svit měsíčka; \\
i zaplává rudoskvěle, \\
jak by západ to sluníčka.
\end{verse}

\begin{verse}
I~vidouc to žena žasne \\
a ke vchodu až pokročí, \\
a zastíníc dlaní oči, \\
hledí v~ono místo jasné. \\
„Bože, jak se to tam svítí!“ \\
Oči rukou si protírá, \\
o~krok blíže se ubírá: \\
„Jak se to tam divně svítí! \\
Co to asi může býti?“ \\
Dále jíti však se bojí, \\
hledíc tam, a venku stojí. \\
A~co váhá a co stojí, \\
v~klenbu patříc neustále, \\
mizí bázeň za pohledem, \\
zvědavost ji pudí předem, \\
a žena se béře dále. \\
Krok za krokem -- a vždy dále \\
mocně ji to jíti pudí; \\
krok za krokem -- a ve skále \\
jen se spící ohlas budí. \\
A~čím dál přichází žena, \\
stále divná roste záře. \\
Již již končí se sklepení; \\
avšak žena omráčena \\
rukou zakrývá si tváře, \\
přímo patřit možné není. \\
Vidí, vidí -- co zde vidí, \\
kdy to viděl který z~lidí? \\
Tolik krásy, tolik blesku \\
mní uzříti jen v~nebesku!
\end{verse}

\begin{verse}
Dvéře tu jsou otevřeny \\
do nejskvělejšího sálu; \\
zlatem jen se svítí stěny, \\
strop rubíny vyložený, \\
pod ním sloupy ze křišťálu. \\
Z~obojí pak strany dveří \\
na podlaze mramorové -- \\
kdo neviděl, neuvěří -- \\
hoří, hoří dva ohňové; \\
dva ohňové tuto hoří, \\
nic jich blesku neumoří: \\
nade stříbrem po levici \\
lunou oheň vzhůru plane, \\
nade zlatem po pravici \\
sluncem pláti neustane. \\
Planou ohně, jizba plane, \\
obalena září jasnou; \\
a dokud tu poklad stane, \\
plamenové nevyhasnou, \\
nic jich blesku neumoří.
\end{verse}

\begin{verse}
Na prahu tu žena stojí, \\
celá stojí oslepena; \\
očí pozdvihnout se bojí, \\
nemůž zříti do plamena. \\
Na levici dítě nese, \\
pravou levé mne si oko; \\
a když trochu ohlídne se, \\
osmělí a vzpamatuje, \\
vzdychne sobě přehluboko \\
a tak v~duši své rokuje: \\
„Milý bože, co já zkusím \\
na tom světě nouze, hladu! \\
Bídně život chránit musím -- \\
a zde tolik těch pokladů! \\
Tolik stříbra, tolik zlata \\
v~podzemní tu leží skrejši! \\
Jenom hrstku z~té hromady -- \\
a já byla bych bohata, \\
a byla bych nejšťastnější, \\
já i moje dítě tady!"
\end{verse}

\begin{verse}
A~co myslí a co stojí, \\
ohroženější se stane; \\
svatým křížem se ozbrojí \\
a jde, kde to běle plane. \\
Jde a stříbra kousek zdvihne, \\
avšak zase tam položí; \\
zdvihne zas a je prohlíží, \\
jeho blesk a jeho tíži -- \\
a zdali je zas položí? \\
Ne, již v~klíně jí se mihne. \\
A~zdařením tím smělejší: \\
„Jistě toto prst je boží, \\
poklad ukázal mi v~skrejši, \\
chce, bych byla oblažena: \\
i zhřešiti bych musela, \\
bych jím pohrdnouti měla!"
\end{verse}

\begin{verse}
Takto k~sobě mluvíc žena, \\
chlapce na zem z~ruky složí, \\
klekne a klín rozestírá, \\
chutě z~hromady nabírá \\
a do klína stříbro skládá: \\
„Jistě toto prst je boží, \\
jenž nás obohatit žádá!“ \\
Béře, béře ze hromady -- \\
klín již plný, sotva vstává, \\
ještě v~šátek sobě dává, \\
tak ji mámí stříbra vnady! \\
A~když již chce odtud jíti: \\
ach, zde ještě pacholete! \\
Jak je ke vší tíži vzíti?
\end{verse}

\begin{verse}
Pacholátko již dvouleté; \\
vysypati zase štěstí \\
nezdá se jí dobré býti; \\
obého pak nemůž nésti.
\end{verse}

\begin{verse}
A~hle, stříbro matka nese! \\
Dítě se tu na ni třese: \\
„Mama!“ volá, „mama, mama!“ \\
chytajíc ji ručinkama. \\
„Mlč, synáčku, mlč, mlč, hochu, \\
počkej tuto jenom trochu, \\
hned tu bude zase mama!“
\end{verse}

\begin{verse}
A~již běží, síní běží, \\
již i síně za ní leží; \\
přes potok, po stráni k~lesu \\
spěchá žena ve svém plesu. \\
A~než malá ušla chvíle, \\
prázdná zpátky zas pospíchá; \\
a ve potu, sotva dýchá, \\
stane zase již u~cíle.
\end{verse}

\begin{verse}
A~jak vítr zlehka věje, \\
z~kostela slyšeti pění: \\
v~kůru tam se právě pěje \\
Krista Pána umučení.
\end{verse}

\begin{verse}
A~jak síní v~jizbu spěje: \\
„Haha, mama, haha, mama!“ \\
radostně se dítě směje, \\
potleskujíc ručinkama. \\
Nedbátě však matka na to, \\
běžíc ve stranu protější: \\
kovu blesk je jí milejší, \\
z~kovů nejmilejší zlato. \\
Klekne a klín rozestírá, \\
chutě z~hromady nabírá \\
a do klína zlato skládá. \\
Klín již plný, sotva vstává -- \\
ještě v~šátek sobě dává! \\
Ó jak jí tu srdce skáče, \\
jak je štěstí svému ráda!
\end{verse}

\begin{verse}
A~když zlato matka nese, \\
dítě se tu na ni třese, \\
třese a žalostně pláče: \\
„Mama, mama, ach, ach, mama!“ \\
chytajíc ji ručinkama. \\
„Mlč, synáčku, mlč, mlč, hochu, \\
počkej jenom ještě trochu.“ \\
A~k~dítěti se nakloní \\
a do klína rukou sáhne, \\
dva peníze ven vytáhne, \\
o~peníz penízem zvoní: \\
„Hle hleď, co to má maminka! \\
Cincin! slyšíš, jak to cinká?“ \\
Avšak dítě stále pláče -- \\
jí radostí srdce skáče.
\end{verse}

\begin{verse}
A~do klína opět sáhne, \\
plnou zlata hrst vytáhne, \\
vloží dítěti do klínka: \\
„Hle hleď, co ti dá maminka!
\end{verse}

\begin{verse}
Mlč, synáčku, mlč, mlč, hochu: \\
cincin! poslyš, jak to cinká! \\
Počkej jenom ještě trochu, \\
hned se vrátí zas maminka. \\
Hrej si pěkně, hrej, děťátko, \\
počkej ještě jen drobátko.“
\end{verse}

\begin{verse}
A~již běží, síní běží, \\
na dítě se neohlíží; \\
a již síně za ní leží, \\
již se ku potoku blíží; \\
přes potok, po stráni v~plesu \\
drahý poklad nese k~lesu, \\
a již stojí s~ním před chýží.
\end{verse}

\begin{verse}
„Hoj, ty chýže, sprostá chýže, \\
brzy měj se dobře tady! \\
Co mě k~tobě nyní víže? \\
Nenalézám v~tobě vnady! \\
Půjdu pryč z~těch tmavých lesů, \\
z~té otcovské střechy chudé; \\
jinde štěstí své ponesu, \\
jinde moje bydlo bude! \\
Půjdu, půjdu z~toho kraje, \\
radostná mi odtud cesta, \\
půjdu, když mi štěstí zraje, \\
do velkého půjdu města; \\
koupím sobě země, hrady, \\
co paní mě budou ctíti: \\
měj se dobře, chýžko, tady, \\
nebudu já v~tobě žíti!
\end{verse}

\begin{verse}
Nejsem již ta chudá vdova, \\
péči nesouc v~noci ve dne: \\
ejhle v~klínu“ -- na ta slova \\
s~potěšením tam pohledne. -- \\
Ó kéž byla nepohledla! \\
Leknutím tu celá zbledla, \\
leknutím se třese celá, \\
div na místě neomdlela. \\
Vidí, vidí -- ha, co vidí, \\
sama tomu sotva věří! \\
Do zpukřelých vrazí dveří, \\
vrazí, kde truhlice byla, \\
v~kterou stříbro uložila. \\
Strhne víko -- ha, co vidí! \\
Pro vši víru dobrých lidí, \\
jaká opět nová rána! \\
Místo stříbra jen -- kamení, \\
v~šátku pak a ve svém klínu, \\
ó přehrozné to mámení, \\
místo zlata -- samou hlínu! \\
Čáka všecka rozšlapána! -- --
\end{verse}

\begin{verse}
Nehodnatě štěstí byla, \\
požehnání neužila.
\end{verse}

\begin{verse}
II
\end{verse}

\begin{verse}
A~když takto rozdrceně \\
s~bolestí tu ztrátu nese, \\
probodne to srdce ženě, \\
vzkřikne s~hrůzou vyděšeně, \\
vzkřikne, a se chýže třese: \\
„Ach dítě, mé dítě drahé! \\
Dítě drahé -- drahé -- drahé!“ \\
zahučelo v~hustém lese.
\end{verse}

\begin{verse}
A~ve hrozném předtušení \\
běží žena -- ach neběží, \\
letí, letem ptáka letí, \\
lesem, strání ji viděti, \\
tam, kde klamné našla jmění, \\
k~vršku, na něm kostel leží.
\end{verse}

\begin{verse}
Od kostela větřík věje, \\
copak neslyšeti pění? -- \\
Krista Pána umučení \\
v~kůru tam se již nepěje.
\end{verse}

\begin{verse}
A~když přišla ke sklepení, \\
haha, jaké pohledění! \\
Haha, z~divokého klestu \\
tři sta kroků od kostela \\
veliký ční kámen v~cestu! \\
A~kde síně? -- Ta zmizela! \\
Zmizela, i v~cestě skála, \\
jak by nikdy zde nestála.
\end{verse}

\begin{verse}
Ha, jak se tu žena leká, \\
jak se děsí, volá, \\
hledá, jak po tom pahorku těká, \\
těmi klesty, na smrt bledá! \\
Ha, ty zraky zoufanlivé, \\
ústa siná nad mrtvolu! \\
Hle, jak přes to křoví divé \\
běží -- pádí tamto k~dolu!
\end{verse}

\begin{verse}
Běda, běda! Zde to není! \\
Tělo klestím rozervané, \\
nohy trním probodané -- \\
darmo všecko klopotění, \\
vchodu již nalézti není!
\end{verse}

\begin{verse}
A~znovu se žena děsí, \\
úzkost hrozná ji uchvátí: \\
„Ach, kdo mně mé dítě vrátí! \\
Ach mé dítě, kde jsi, kde jsi?!“ -- \\
„Tu pod zemí jsem, hluboko!“ \\
hlas tichounký větrem šumí, \\
nespatří mne žádné oko, \\
ucho mi neporozumí. \\
Blaze tu pod zemí, blaze, \\
beze jídla, beze pití, \\
na mramorové podlaze, \\
ryzí zlato v~klínku míti!
\end{verse}

\begin{verse}
Noc a den se nestřídají \\
nikdy nejdou spát očinka: \\
hraji si tu pěkně, hraji -- \\
cincin! slyšíš, jak to cinká?
\end{verse}

\begin{verse}
Avšak žena znovu hledá -- \\
darmo, a znovu se děsí, \\
zoufale se na zem vrhá, \\
vlasy sobě z~hlavy trhá, \\
zkrvavena, na smrt bledá: \\
„Ach běda mi! Běda, běda! \\
Ach mé dítě, kde jsi, kde jsi? \\
Kde tě najdu, dítě drahé?! \\
Dítě drahé -- drahé -- drahé!“ \\
blízkými to hučí lesy.
\end{verse}

\begin{verse}
III
\end{verse}

\begin{verse}
Mine den, i druhý mine, \\
dnové v~týden se obrátí, \\
z~týdnů měsíc se vyvine, \\
a i léto počne pláti.
\end{verse}

\begin{verse}
Na pahorku mezi buky \\
kostelíček s~věží nízkou; \\
co den znějí zvonka zvuky \\
hájem a sousední vískou. \\
Tu nahoře, když se zrána \\
ke mši zvonečkem pozvoní, \\
přede stánkem nebes pána \\
zbožný rolník čelo kloní.
\end{verse}

\begin{verse}
Aj, kdo zná ji, tu osobu \\
se sklopenou k~zemi tváří? \\
Svíce zhasly na oltáři, \\
ona klečí po tu dobu. \\
Zdáť se, ani že nedýše -- \\
líce a rty zesinalé -- \\
ach, to se tak modlí tiše! \\
Kdo to? -- Nevím, tuším ale. \\
Když po svaté však oběti \\
chrámové se zamknou dvéře, \\
těmi buky ji viděti, \\
ana se z~pahorku béře. \\
Béře, béře se pomálu \\
stezkou vinoucí se v~klestu \\
po šedivou tamo skálu, \\
kde ční kámen velký v~cestu. \\
Tu si vzdychne přehluboko \\
a do dlaně čelo sklopí: \\
„Ach mé dítě!“ -- a již oko \\
v~slzách kanoucích se topí.
\end{verse}

\begin{verse}
Nešťastná to z~chýže žena, \\
vždycky smutná, vždycky bledá, \\
vždycky těžce zamyšlena; \\
od rána a do soumraku \\
nikdy jasno v~jejím zraku, \\
v~noci pak žel spáti nedá. \\
A~když opět na úsvitě \\
traplivé opouští lože: \\
„Ach mé dítě, drahé dítě, \\
ach běda mi, běda, běda! \\
Odpusť milostivý Bože!“
\end{verse}

\begin{verse}
Uplynulo léto celé, \\
jeseň, zima uplynula -- \\
nezmírněno v~srdci žele, \\
slza v~oku nezhynula. \\
I~když výše slunce stálo, \\
rozehřávši zemi znova, \\
úst k~úsměchu nerozhřálo, \\
stáleť ještě pláče vdova.
\end{verse}

\begin{verse}
IV
\end{verse}

\begin{verse}
A~slyš, shůry mezi buky, \\
z~kostelíčka s~věží nízkou, \\
rachotící slyšet zvuky \\
hájem a sousední vískou. \\
A~hle, vzhůru k~boží slávě \\
běží z~vísky zástup hojný, \\
veský lid to bohabojný -- \\
a dnes Velký pátek právě.
\end{verse}

\begin{verse}
Jemně jarní větřík věje, \\
větrem pak slyšeti pění: \\
v~kostele se zase pěje \\
Krista Pána umučení.
\end{verse}

\begin{verse}
A~tou strání ku potoku \\
žena od lesa se blíží. \\
Co zdržuje dnes ji v~kroku? -- \\
Ach, památka dne a roku \\
hořem kroky její tíží! \\
Blíží, blíží se znenáhla, \\
a již skály té dosáhla.
\end{verse}

\begin{verse}
A~hle, co se jeví oku? \\
Tu, kde z~divokého klestu, \\
od kostela tři sta kroků, \\
veliký čněl kámen v~cestu, \\
vchodem vršek otevřený, \\
kámen v~cestu postavený, \\
kámen i ta celá skála, \\
jak by tak od věků stála.
\end{verse}

\begin{verse}
A~žena se toho leká, \\
hrůzou se jí vlasy ježí: \\
celou tíží na ní leží \\
zármutek a vina její. \\
I~děsí se -- však nečeká, \\
a ve strachu a v~naději \\
skokem síní známou běží, \\
síní jdoucí pode skálu.
\end{verse}

\begin{verse}
A~hle, dvéře otevřeny \\
do nejskvělejšího sálu; \\
zlatem jen se svítí stěny, \\
strop rubíny vyložený, \\
pod ním sloupy ze křišťálu. \\
A~z~obojí strany dveří \\
na podlaze mramorové \\
plápolají dva ohňové: \\
nade stříbrem po levici \\
lunou oheň vzhůru plane, \\
nade zlatem po pravici \\
sluncem pláti nepřestane.
\end{verse}

\begin{verse}
A~žena se s~hrůzou blíží \\
a ve strachu a v~naději \\
tu po jizbě se ohlíží. \\
Snad ji vábí stříbro, zlato? -- \\
Ach, již ona nedbá na to! \\
„Haha, mama, haha, mama!“ \\
Ejhle dítě, dítě její, \\
po celý rok oplakané, \\
potleskuje ručinkama!
\end{verse}

\begin{verse}
Ale v~ženě není dechu, \\
a hrůzou se celá třese, \\
a ve zoufanlivém spěchu \\
chopíc dítě do náručí, \\
dlouhou síní odtud nese.
\end{verse}

\begin{verse}
A~třesk, třesk! huhu! to hučí \\
jí v~patách ve vrchu klíně; \\
praskot hrozný, vichr skučí, \\
zem se třese, hluk a lomoz -- \\
jí v~patách se boří síně! \\
„Ach, rodičko boží, pomoz!“ \\
v~úzkosti tu volá žena, \\
zpět pohlédnouc poděšena.
\end{verse}

\begin{verse}
A~hle! jaká zase změna! \\
Ticho všecko, a tu z~klestu \\
veliký ční kámen v~cestu; \\
vše jak jindy spořádáno, \\
po vchodu památky není: \\
právě nyní dozpíváno \\
Krista Pána umučení.
\end{verse}

\begin{verse}
Ale v~ženě není dechu, \\
a hrůzou se celá třese, \\
a ve zoufanlivém spěchu \\
dítě svoje odtud nese, \\
nese a na ňadra tlačí, \\
jako by se o~ně bála; \\
běží, sotva dech jí stačí, \\
ač daleko za ní skála; \\
běží, aniž se ohlíží, \\
tam tou strání blíže lesu, \\
a ve strachu a ve plesu \\
stane v~chudé lesní chýži.
\end{verse}

\begin{verse}
Ó jaké tu vzdává vroucí \\
bohu svému žena díky! \\
Vizte slzy ty kanoucí! \\
Jak to dítě k~sobě vine, \\
líbá čelo, ručky, rtíky, \\
a zas k~ňadrám je přitiská, \\
jak celá v~rozkoi plyne!
\end{verse}

\begin{verse}
A~hle, co se v~klínku blýská? \\
Co to znělo? -- Ryzí zlato! \\
To zlato, jež loni byla, \\
aby dítě si pohrálo, \\
jemu v~klínek položila. \\
Avšak ženu vábí málo, \\
co ji tolik hoře stálo! \\
Stáloť ji, ach, slzí mnoho; \\
leč děkujíc bohu za to, \\
touže drahé tiskne děcko. \\
Hořceť zakusila toho: \\
žetě velmi málo zlato, \\
avšak dítě nade všecko!
\end{verse}

\begin{verse}
V~\end{verse}

\begin{verse}
Dávno kostelíček zbořem; \\
umlkly již zvonka zvuky; \\
a kde někdy stály buky, \\
sotva jaký hnije kořen.
\end{verse}

\begin{verse}
Stařec mnoho pamatuje, \\
mnohoť i dozrálo hrobu, \\
avšak lid si ukazuje \\
ještě místa po tu dobu.
\end{verse}

\begin{verse}
A~když večer pohromadě \\
mládež za mrazu sedává, \\
rád stařeček povídává \\
o~vdově a o~pokladě.
\end{verse}