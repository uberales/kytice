\PoemTitle{Kytice}
\begin{verse}
Zemřela matka a do hrobu dána, \\
siroty po ní zůstaly; \\
i přicházely každičkého rána \\
a matičku svou hledaly.
\end{verse}

\begin{verse}
I~zželelo se matce milých dítek; \\
duše její se vrátila \\
a vtělila se v~drobnolistí kvítek, \\
jím mohylu svou pokryla.
\end{verse}

\begin{verse}
Poznaly dítky matičku po dechu, \\
poznaly ji a plesaly; \\
a prostý kvítek, v~něm majíc útěchu, \\
mateřídouškou nazvaly. --
\end{verse}

\begin{verse}
Mateřídouško vlasti naší milé, \\
vy prosté naše pověsti, \\
natrhal jsem tě na dávné mohyle -- \\
komu mám tebe přinésti?
\end{verse}

\begin{verse}
Ve skrovnou já tě kytici zavážu, \\
ozdobně stužkou ovinu; \\
do šírých zemí cestu ti ukážu, \\
kde příbuznou máš rodinu.
\end{verse}

\begin{verse}
Snad se najde dcera mateřina, \\
jí mile dech tvůj zavoní; \\
snad že i najdeš některého syna, \\
jenž k~tobě srdce nakloní!
\end{verse}