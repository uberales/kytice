\PoemTitle{Polednice}
\begin{verse}
U~lavice dítě stálo, \\
z~plna hrdla křičelo. \\
„Bodejž jsi jen trochu málo, \\
ty cikáně, mlčelo!
\end{verse}

\begin{verse}
Poledne v~tom okamžení, \\
táta přijde z~roboty: \\
a mně hasne u~vaření \\
pro tebe, ty zlobo, ty!
\end{verse}

\begin{verse}
Mlč, hle husar a kočárek -- \\
hrej si -- tu máš kohouta!“ -- \\
Než kohout, vůz i husárek \\
bouch, bác! letí do kouta.
\end{verse}

\begin{verse}
A~zas do hrozného křiku -- \\
„I bodejž tě sršeň sám! -- \\
že na tebe, nezvedníku, \\
polednici zavolám!
\end{verse}

\begin{verse}
Pojď si proň, ty polednice, \\
pojď, vem si ho, zlostníka!“ \\
A~hle, tu kdos u~světnice \\
dvéře zlehka odmyká.
\end{verse}

\begin{verse}
Malá, hnědá, tváři divé \\
pod plachetkou osoba; \\
o~berličce, hnáty křivé, \\
hlas -- vichřice podoba!
\end{verse}

\begin{verse}
„Dej sem dítě!“ -- „Kriste Pane, \\
odpusť hříchy hříšnici!“ \\
Div že smrt jí neovane, \\
ejhle tuť -- polednici!
\end{verse}

\begin{verse}
Ke stolu se plíží tiše \\
polednice jako stín: \\
matka hrůzou sotva dýše, \\
dítě chopíc na svůj klín.
\end{verse}

\begin{verse}
A~vinouc je, zpět pohlíží -- \\
běda, běda dítěti! \\
Polednice blíž se plíží, \\
blíž -- a již je v~zápětí.
\end{verse}

\begin{verse}
Již vztahuje po něm ruku -- \\
matka tisknouc ramena: \\
„Pro Kristovu drahou muku!“ \\
klesá smyslů zbavena.
\end{verse}

\begin{verse}
Tu slyš: jedna -- druhá -- třetí -- \\
poledne zvon udeří; \\
klika cvakla, dvéře letí -- \\
táta vchází do dveří.
\end{verse}

\begin{verse}
Ve mdlobách tu matka leží, \\
k~ňadrám dítě přimknuté; \\
matku vzkřísil ještě stěží, \\
avšak dítě -- zalknuté.
\end{verse}