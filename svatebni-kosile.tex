\PoemTitle{Svatební košile}
\begin{verse}
Již jedenáctá odbila, \\
a lampa ještě svítila, \\
a lampa ještě hořela, \\
co nad klekadlem visela.
\end{verse}

\begin{verse}
Na stěně nízké světničky \\
byl obraz boží rodičky, \\
rodičky boží s~děťátkem, \\
tak jako růže s~poupátkem.
\end{verse}

\begin{verse}
A~před tou mocnou světicí \\
viděti pannu klečící: \\
klečela, líce skloněné, \\
ruce na prsa složené; \\
slzy jí z~očí padaly, \\
čelem se ňádra zdvihaly. \\
A~když slzička upadla, \\
v~ty bílé ňádra zapadla.
\end{verse}

\begin{verse}
„Žel bohu, kde můj tatíček? \\
Již na něm roste trávníček! \\
Žel bohu, kde má matička? \\
Tam leží -- podle tatíčka! \\
Sestra do roka nežila, \\
bratra mi koule zabila.
\end{verse}

\begin{verse}
Měla jsem, smutná, milého, \\
život bych dala pro něho! \\
Do ciziny se obrátil, \\
potud se ještě nevrátil. \\
Do ciziny se ubíral, \\
těšil mě, slzy utíral: \\
„Zasej, má milá, zasej len, \\
vzpomínej na mne každý den, \\
první rok přádla hledívej, \\
druhý rok plátno polívej, \\
třetí košile vyšívej: \\
až ty košile ušiješ, \\
věneček z~routy poviješ.“
\end{verse}

\begin{verse}
Již jsem košile ušila, \\
již jsem je v~truhle složila, \\
již moje routa v~odkvětě, \\
a milý ještě ve světě, \\
ve světě šírém, širokém, \\
co kámen v~moři hlubokém. \\
Tři léta o~něm ani sluch \\
živ-li a zdráv -- zná milý bůh!
\end{verse}

\begin{verse}
Maria, panno přemocná, \\
ach budiž ty mi pomocna: \\
vrať mi milého z~ciziny, \\
květ blaha mého jediný; \\
milého z~ciziny mi vrať -- \\
aneb život můj náhle zkrať: \\
u~něho život jarý květ -- \\
bez něho však mě mrzí svět. \\
Maria, matko milosti, \\
buď pomocnicí v~žalosti!“
\end{verse}

\begin{verse}
Pohnul se obraz na stěně -- \\
i vzkřikla panna zděšeně; \\
lampa, co temně hořela, \\
prskla a zhasla docela. \\
Možná, žeť větru tažení, \\
možná i -- zlé že znamení!
\end{verse}

\begin{verse}
A~slyš, na záspí kroků zvuk \\
a na okénko: ťuk, ťuk, ťuk! \\
„Spíš, má panenko, nebo bdíš? \\
Hoj, má panenko, tu jsem již! \\
Hoj, má panenko, co děláš? \\
A~zdalipak mě ještě znáš, \\
aneb jiného v~srdci máš?“
\end{verse}

\begin{verse}
„Ach můj milý, ach pro nebe, \\
tu dobu myslím na tebe; \\
na tě jsem vždycky myslila, \\
za tě se právě modlila!“
\end{verse}

\begin{verse}
„Ho, nech modlení -- skoč a pojď, \\
skoč a pojď a mě doprovoď; \\
měsíček svítí na cestu: \\
já přišel pro svou nevěstu.“
\end{verse}

\begin{verse}
„Ach proboha, ach co pravíš? \\
Kam bychom šli -- tak pozdě již! \\
Vítr burácí, pustá noc, \\
počkej jen do dne -- není moc.“
\end{verse}

\begin{verse}
„Ho, den je noc, a noc je den -- \\
ve dne mé oči tlačí sen! \\
Dřív než se vzbudí kohouti, \\
musím tě za svou pojmouti. \\
Jen neprodlévej, skoč a pojď, \\
dnes ještě budeš moje choť!“
\end{verse}

\begin{verse}
Byla noc, byla hluboká, \\
měsíček svítil z~vysoka \\
a ticho, pusto v~dědině, \\
vítr burácel jedině.
\end{verse}

\begin{verse}
A~on tu napřed -- skok a skok, \\
a ona za ním, co jí krok. \\
Psi houfem ve vsi zavyli, \\
když ty pocestné zvětřili; \\
a vyli, vyli divnou věc: \\
žetě nablízku umrlec!
\end{verse}

\begin{verse}
„Pěkná noc, jasná -- v~tu dobu \\
vstávají mrtví ze hrobů, \\
a nežli zvíš, jsou tobě blíž -- \\
má milá, nic se nebojíš?“
\end{verse}

\begin{verse}
„Což bych se bála? Tys se mnou, \\
a oko boží nade mnou. -- \\
Pověz, můj milý, řekni přec, \\
živ-li a zdráv je tvůj otec? \\
Tvůj otec a tvá milá máť, \\
a ráda-li mě bude znáť?“
\end{verse}

\begin{verse}
„Moc, má panenko, moc se ptáš, \\
jen honem pojď -- však uhlídáš. \\
Jen honem pojď -- čas nečeká, \\
a cesta naše daleká. \\
Co máš, má milá, v~pravici?“
\end{verse}

\begin{verse}
„Nesu si knížky modlicí.“ \\
„Zahoď je pryč, to modlení \\
je těžší nežli kamení! \\
Zahoď je pryč, ať lehce jdeš, \\
jestli mi postačiti chceš.“
\end{verse}

\begin{verse}
Knížky jí vzal a zahodil, \\
a byli skokem deset mil.
\end{verse}

\begin{verse}
A~byla cesta výšinou, \\
skalami, lesní pustinou; \\
a v~rokytí a v~úskalí \\
divoké feny štěkaly; \\
a kulich hlásal pověsti: \\
žetě nablízku neštěstí. --
\end{verse}

\begin{verse}
A~on vždy napřed -- skok a skok, \\
a ona za ním, co jí krok. \\
Po šípkoví a po skalí \\
ty bílé nohy šlapaly; \\
a na hloží a křemení \\
zůstalo krve znamení.
\end{verse}

\begin{verse}
„Pěkná noc, jasná -- v~tento čas \\
mrtví s~živými chodí zas; \\
a nežli zvíš, jsou tobě blíž -- \\
má milá, nic se nebojíš?“
\end{verse}

\begin{verse}
„Což bych se bála? Tys se mnou, \\
a ruka Páně nade mnou. -- \\
Pověz, můj milý, řekni jen, \\
jak je tvůj domek upraven? \\
Čistá světnička? Veselá? \\
A~zdali blízko kostela?“
\end{verse}

\begin{verse}
„Moc, má panenko, moc se ptáš, \\
však ještě dnes to uhlídáš. \\
Jen honem pojď -- čas utíká, \\
a dálka ještě veliká. \\
Co máš, má milá, za pasem?“
\end{verse}

\begin{verse}
„Růženec s~sebou vzala jsem.“
\end{verse}

\begin{verse}
„Ho, ten růženec z~klokočí \\
jako had tebe otočí! \\
Zúží tě, stáhne tobě dech: \\
zahoď jej pryč -- neb máme spěch!“
\end{verse}

\begin{verse}
Růženec popad, zahodil, \\
a byli skokem dvacet mil.
\end{verse}

\begin{verse}
A~byla cesta nížinou, \\
přes vody, luka, bažinou; \\
a po bažině, po sluji \\
modrá světélka laškují: \\
dvě řady, devět za sebou, \\
jako když s~tělem k~hrobu jdou; \\
a žabí havěť v~potoce \\
pohřební píseň skřehoce. --
\end{verse}

\begin{verse}
A~on vždy napřed -- skok a skok, \\
a jí za ním již slábne krok. \\
Ostřice dívku ubohou \\
břitvami řeže do nohou; \\
a to kapradí zelené \\
je krví její zbarvené.
\end{verse}

\begin{verse}
„Pěkná noc, jasná -- v~tu dobu \\
spěchají živí ke hrobu; \\
a nežli zvíš, jsi hrobu blíž -- \\
má milá, nic se nebojíš?“
\end{verse}

\begin{verse}
„Ach nebojím, vždyť tys se mnou, \\
a vůle Páně nade mnou! \\
Jen ustaň málo v~pospěchu, \\
jen popřej málo oddechu. \\
Duch slábne, nohy klesají, \\
a k~srdci nože bodají!“
\end{verse}

\begin{verse}
„Jen pojď a pospěš, děvče mé, \\
však brzo již tam budeme. \\
Hosté čekají, čeká kvas, \\
a jako střela letí čas. -- \\
Co to máš na té tkaničce \\
na krku na té tkaničce?“
\end{verse}

\begin{verse}
„To křížek po mé matičce.“
\end{verse}

\begin{verse}
„Hoho, to zlato proklaté \\
má hrany ostře špičaté! \\
Bodá tě -- a mě nejinak, \\
zahoď to, budeš jako pták!
\end{verse}

\begin{verse}
Křížek utrh a zahodil, \\
a byli skokem třicet mil. --
\end{verse}

\begin{verse}
Tu na planině široké \\
stavení stojí vysoké; \\
úzká a dlouhá okna jsou, \\
a věž se zvonkem nad střechou.
\end{verse}

\begin{verse}
„Hoj, má panenko, tu jsme již! \\
Nic, má panenko, nevidíš?“
\end{verse}

\begin{verse}
„Ach proboha, ten kostel snad?“
\end{verse}

\begin{verse}
„To není kostel, to můj hrad!“
\end{verse}

\begin{verse}
„Ten hřbitov -- a těch křížů řad?“
\end{verse}

\begin{verse}
„To nejsou kříže, to můj sad! \\
Hoj, má panenko, na mě hleď \\
a skoč vesele přes tu zeď!“
\end{verse}

\begin{verse}
„Ó nech mne již, ó nech mne tak! \\
Divý a hrozný je tvůj zrak; \\
tvůj dech otravný jako jed, \\
a tvoje srdce tvrdý led!“
\end{verse}

\begin{verse}
„Nic se, má milá, nic neboj! \\
Veselo u~mne, všeho hoj: \\
masa dost -- ale bez krve, \\
dnes bude jinak poprvé! -- \\
Co máš v~uzlíku, má milá?"
\end{verse}

\begin{verse}
"Košile, co jsem ušila.“
\end{verse}

\begin{verse}
„Netřeba jich víc nežli dvě: \\
ta jedna tobě, druhá mně.“
\end{verse}

\begin{verse}
Uzlík jí vzal a s~chechtotem \\
přehodil na hrob za plotem. \\
„Nic ty se neboj, na mě hleď \\
a skoč za uzlem přes tu zeď.“
\end{verse}

\begin{verse}
„Však jsi ty vždy byl přede mnou, \\
a já za tebou cestou zlou; \\
však jsi byl napřed po ten čas: \\
skoč a ukaž mi cestu zas!“
\end{verse}

\begin{verse}
Skokem přeskočil ohradu, \\
nic nepomyslil na zradu; \\
skočil do výše sáhů pět -- \\
jí však již venku nevidět: \\
jenom po bílém obleku \\
zablesklo se jest v~útěku, \\
a schrána její blízko dost -- \\
nenadál se zlý její host!
\end{verse}

\begin{verse}
Stojíť tu, stojí komora: \\
nizoučké dvéře -- závora; \\
zavrzly dvéře za pannou \\
a závora jí ochranou. \\
Stavení skrovné, bez oken, \\
měsíc lištami šeřil jen; \\
stavení pevné jako klec, \\
a v~něm na prkně -- umrlec.
\end{verse}

\begin{verse}
Hoj, jak se venku vzmáhá hluk, \\
hrobových oblud mocný pluk; \\
šumí a kolem klapají \\
a takto píseň skuhrají:
\end{verse}

\begin{verse}
„Tělu do hrobu přísluší, \\
běda, kdos nedbal o~duši!“ \\
A~tu na dvéře: buch, buch, buch! \\
burácí zvenčí její druh: \\
„Vstávej, umrlče, nahoru, \\
odstrč mi tam tu závoru!“
\end{verse}

\begin{verse}
A~mrtvý oči otvírá, \\
a mrtvý oči protírá, \\
sbírá se, hlavu pozvedá \\
a půlkolem se ohlédá.
\end{verse}

\begin{verse}
„Bože svatý, rač pomoci, \\
nedej mne ďáblu do moci! -- \\
Ty mrtvý, lež a nevstávej, \\
pánbůh ti pokoj věčný dej!“
\end{verse}

\begin{verse}
A~mrtvý hlavu položiv, \\
zamhouřil oči jako dřív. --
\end{verse}

\begin{verse}
A~tu poznovu -- buch, buch, buch! \\
silněji tluče její druh: \\
„Vstávej, umrlče, nahoru, \\
otevři mi svou komoru!“
\end{verse}

\begin{verse}
A~na ten hřmot a na ten hlas \\
mrtvý se zdvihá z~prkna zas \\
a rámě ztuhlé naměří \\
tam, kde závora u~dveří.
\end{verse}

\begin{verse}
„Spas duši, Kriste Ježíši, \\
smiluj se v~bídě nejvyšší! -- \\
Ty mrtvý, nevstávej a lež; \\
pánbůh tě potěš -- a mne též!“
\end{verse}

\begin{verse}
A~mrtvý zas se položiv, \\
natáhnul údy jako dřív. --
\end{verse}

\begin{verse}
A~znova venku: buch, buch, buch! \\
a panně mizí zrak i sluch! \\
Vstávej, umrlče, hola hou, \\
a podej mi sem tu živou!
\end{verse}

\begin{verse}
Ach běda, běda děvčeti! \\
Umrlý vstává potřetí \\
a velké, kalné své oči \\
na poloumrtvou otočí. \\
„Maria Panno, při mně stůj, \\
u~syna svého oroduj! \\
Nehodně jsem tě prosila: \\
ach odpusť, co jsem zhřešila! \\
Maria, matko milosti, \\
z~té moci zlé mě vyprosti.“
\end{verse}

\begin{verse}
A~slyš, tu právě nablízce \\
kokrhá kohout ve vísce; \\
a za ním, co ta dědina, \\
všecka kohoutí družina.
\end{verse}

\begin{verse}
Tu mrtvý, jak se postavil, \\
pádem se na zem povalil, \\
a venku ticho -- ani ruch: \\
zmizel dav, i zlý její druh. --
\end{verse}

\begin{verse}
Ráno když lidé na mši jdou, \\
v~úžasu státi zůstanou: \\
hrob jeden dutý nahoře, \\
panna v~umrlčí komoře, \\
a na každičké mohyle \\
útržek z~nové košile. --
\end{verse}

\begin{verse}
Dobře ses, panno, radila, \\
na boha že jsi myslila \\
a druha zlého odbyla! \\
Bys byla jinak jednala, \\
zle bysi byla skonala: \\
tvé tělo bílé, spanilé, \\
bylo by co ty košile!
\end{verse}